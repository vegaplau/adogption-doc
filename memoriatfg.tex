\documentclass[a4paper, 12pt]{article}

% Título, autor y fecha
\title{Adogption}
\date{2024-05-10}
\author{Laura Vega Palacios}

% Paquetes
\usepackage[utf8]{inputenc}
\usepackage{subcaption}
\usepackage[table]{xcolor} % Cargado una sola vez
\usepackage{caption}
\usepackage{float}
\usepackage{graphicx} % Cargado una sola vez
\usepackage{epsfig}
\usepackage{multirow}
\usepackage{colortbl}
\usepackage{array} 
\usepackage[parfill]{parskip}
\setlength{\parindent}{30pt}
\usepackage{imakeidx}
\makeindex[columns=3, title=Indice]
\usepackage{hyperref}
\hypersetup{
    colorlinks=true,
    linkcolor=blue,
    filecolor=magenta,      
    urlcolor=cyan,
    pdftitle={Adogption},
    pdfpagemode=FullScreen,
}
\usepackage{geometry}
\usepackage{enumitem}
\usepackage{tocloft}

% Setup
\setcounter{section}{0}
\providecommand{\keywords}[1]{\textbf{\textit{Palabras clave:}} #1}
\renewcommand{\baselinestretch}{1}
\renewcommand{\tableofcontents}{Índice}
\renewcommand{\listtablename}{Índice de tablas}
\renewcommand{\listfigurename}{Índice de figuras}
\captionsetup[figure]{}
\captionsetup[table]{labelformat=empty}
\def\@citex[#1]#2{\textcolor{blue}{\@cite{\@cite@ofmt}{#2\if@tempswa , #1\fi}}}

% Empieza el documento
\begin{document}


% Portada
\begin{titlepage}
	\pagestyle{plain}
	\centering
	{\includegraphics[width=1\textwidth]{logoUGR.png}\par}
	{\bfseries\LARGE Universidad de Granada \par}
	{\scshape\Large Ingeniería Informática \par}
	\vspace{0.5cm}
	{\itshape\Large Trabajo fin de grado \par}
	{\scshape\Huge Desarrollo de Aplicación Móvil para Adopción Canina \par}
	\vfill
	{\Large Autora \par}
	{\Large Laura Vega Palacios\par}

	{\Large Tutor \par}
	{\Large Juan José Escobar Pérez\par}
	\vfill
	{\Large Junio de 2024 \par}
\end{titlepage} 

% Contra portada
\newpage
\thispagestyle{empty}
\mbox{}

% Resumen
\newpage
\pagestyle{plain}
\section*{Resumen}
A día de hoy, comenzar un proceso de adopción puede ser algo tedioso debido a lo descentralizada que está toda la información relacionada con las protectoras y refugios. Una persona que quiera adoptar algún animal normalmente va a necesitar consultar diversas plataformas o redes para encontrar una protectora cercana o un animal que se adapte a su estilo de vida.

El desarrollo de este \textit{Trabajo de Fin de Grado} tiene como finalidad la planificación y desarrollo de una aplicación móvil que permita la gestión de adopciones, la información de las protectoras y facilite la comunicación entre los adoptantes y las protectoras. En esta aplicación, se podrán dar de alta protectoras o refugios, con sus correspondientes datos y ubicación. Las protectoras o refugios que estén verificadas por los administradores tendrán la capacidad de subir a la plataforma todos los caninos, que aparecerán en las listas de adopción o acogida dependiendo de lo que se haya marcado para cada uno. También se podrán dar de alta usuarios que busquen adoptar o acoger algún can y se incluirán formularios para dar de alta o editar la información de ellos, con distintos apartados para cumplimentar todos los datos requeridos como la descripción, la raza, la edad, el peso, etc. Los usuarios también dispondrán de formularios para darse de alta y editar su información. La aplicación utilizará un sistema de geolocalización para mostrar a los usuarios resultados dentro de la aplicación y su distancia. Estos resultados se podrán ver en un mapa insertado en la misma gracias a la inclusión de barras de búsqueda y filtros rápidos para acotar resultados según los requisitos del usuario. Por otro lado, la aplicación incluirá un sistema de mensajería para facilitar la comunicación tanto entre usuarios y protectoras como con los administradores. Este permitirá el envío de mensajes de texto e imágenes. Se añadirá, además, un sistema de favoritos y la posibilidad de compartir caninos de la aplicación de forma externa a través de un enlace. Finalmente la aplicación contendrá una página de información legal y de contacto a disposición del usuario.

% Pagina en blanco
\newpage
\pagestyle{plain}
\thispagestyle{empty}
\mbox{}

% Abstract
\newpage
\pagestyle{plain}
\section*{Abstract}
As of today, starting an adoption process can be tedious due to the decentralized nature of all the information related to shelters and rescue organizations. A person looking to adopt an animal usually needs to consult various platforms or networks to find a nearby shelter or an animal that fits their lifestyle.

The purpose of this \textit{Final Degree Project} is to plan and develop a mobile application that facilitates adoption management, provides information about shelters, and eases communication between adopters and shelters. In this application, shelters or rescue organizations can register with their corresponding data and location. Shelters or rescues verified by administrators will have the ability to upload all the dogs to the platform, which will appear in adoption or foster lists depending on the status marked for each dog. Users looking to adopt or foster a dog can also register, and forms will be included to add or edit their information, with different sections to fill in all required data such as description, breed, age, weight, etc. Users will also have forms to register and edit their information. The application will use a geolocation system to show users results within the application and their distance. These results can be viewed on an embedded map thanks to the inclusion of search bars and quick filters to narrow down results according to the user's requirements. Additionally, the application will include a messaging system to facilitate communication between users and shelters as well as with administrators. This will allow the sending of text messages and images. Furthermore, a favorites system will be added, along with the ability to share dogs from the application externally through a link. Finally, the application will contain a legal information and contact page available to the user.

% Pagina en blanco
\newpage
\thispagestyle{empty}
\mbox{}

% Agradecimientos
\newpage
\section*{Agradecimientos}
\begin{center} 
\vspace*{\fill}
A mis tíos, que me dieron la oportunidad de estudiar y seguir adelante.
\vspace*{\fill}
\end{center} 

% Pagina en blanco
\newpage
\thispagestyle{empty}
\mbox{}
% Pagina en blanco
\newpage
\thispagestyle{empty}
\mbox{}

% Indice
\tableofcontents
\listoftables
\listoffigures

% Pagina en blanco
\newpage
\thispagestyle{empty}
\mbox{}

% Introduccion
\newpage
\section{Introducción}

% Motivación
\subsection{Motivación}
Todos los años, numerosas familias se animan a incluir una mascota en su círculo. Sin embargo, miles de mascotas son a su vez abandonadas por muchas de estas familias en todo el mundo. Existen estudios \cite{affinity} que indican que casi 300.000 perros y gatos fueron recogidos sólo en 2022. Estos datos hacen saltar las alarmas de muchas de las asociaciones que luchan por el bienestar y los derechos de los animales. De todos los animales que son abandonados, muchos permanecen en las calles durante el resto de su vida. Otros son recogidos por las autoridades pertinentes y terminan en protectoras o refugios a la espera de encontrar otra familia. Existen organizaciones sin ánimo de lucro que se encargan de ayudar a muchas de las mascotas que están en las calles. Algunas de estas organizaciones son públicas y subvencionadas por el estado. Para garantizar el bienestar y la protección de los animales, en España se publicó una ley \cite{ley} en la que se tratan principalmente los siguientes puntos:
\begin{itemize}[noitemsep]
\item \textbf{Principios generales}
	\begin{itemize}[noitemsep]
	\item Reconocimiento de los animales como seres dotados de sensibilidad.
	\item Fomento de la adopción en lugar de la compra de animales.
	\item Prohibición de prácticas que causen sufrimiento o estrés innecesario a los animales.
	\item Concienciar acerca del bienestar y respeto animal.
	\end{itemize}
\item \textbf{Responsabilidad y tenencia responsable de animales de compañía}
	\begin{itemize}[noitemsep]
	\item Establecimiento de requisitos mínimos de bienestar, cuidado y alojamiento.
	\item Obligación de identificación y registro de los animales de compañía.
	\item Establecimiento de las obligaciones de los propietarios en cuanto a la tenencia responsable, incluyendo la alimentación, hogar y atención veterinaria.
	\item Prohibición de mantener animales en condiciones inadecuadas o de privarles de cuidados esenciales.
	\end{itemize}
\item \textbf{Cría y comercio de animales}
	\begin{itemize}[noitemsep]
	\item Regulación estricta de la cría y comercio de animales de compañía para evitar la explotación y el maltrato.
	\item Obligación de los criadores y comerciantes de cumplir con requisitos específicos de bienestar animal.
	\item Prohibición de la cría indiscriminada y la venta de animales en tiendas físicas, salvo excepciones debidamente justificadas.
	\end{itemize}
\item \textbf{Concienciación y medidas del estado contra abandonos y abusos}
	\begin{itemize}[noitemsep]
	\item Promoción de la educación, el respeto y protección de los animales.
	\item Campañas de conciencación pública sobre el bienestar animal y la tenencia responsable.
	\item Tipificación de conductas de maltrato y abandono como infracciones administrativas o penales, además del establecimiento de sanciones proporcionales a la gravedad de las infracciones.
	\item Obligación de las administraciones públicas de establecer y mantener refugios y centros de acogida para animales abandonados.
	\item Creación de un registro nacional de animales de compañía y establecimientos relacionados con ellos.
	\end{itemize}
\end{itemize}

Esta ley ha sido un avance muy significativo en la legislación española en materia de protección animal, ayudando a crear un entorno más respetuoso y justo para los animales.

A pesar de que la compra de animales en España es legal, muchas organizaciones de bienestar animal y de los derechos de los animales recomiendan optar por la adopción en lugar de la compra. Esta recomendación tiene su origen en lo habitual que es encontrar criaderos donde los animales no disponen de las condiciones necesarias para vivir adecuadamente. Algunos de estos criaderos carecen de cuidados veterinarios, de una alimentación adecuada y de higiene. En algunos artículos \cite{petalatino} podemos encontrar más información sobre esta problemática. Si se opta finalmente por la compra, se recomienda visitar los criaderos a los que se va a comprar el animal para poder garantizar que no se están fomentando criaderos ilegales y que cumplen con las ley de bienestar animal.


Cuando se opta por la adopción, es necesario cumplir un proceso que puede variar según la organización a la que se acuda. Hay ciertos pasos que son comunes después de escoger una organización:

\begin{itemize}[noitemsep]
\item \textbf{Elección del animal:} en este paso, se deberán tener en cuenta las necesidades de la mascota para determinar cuál se adapta mejor al estilo de vida del hogar donde va a residir. Lo habitual en esta etapa es interactuar con diferentes mascotas candidatas a ser adoptadas.
\item \textbf{Solicitud de adopción:} es común que se tenga que cumplimentar un formulario y realizar una entrevista posterior para asegurar que el animal va a ir a un hogar adecuado.
\item \textbf{Evaluación del hogar:} se organiza una visita al hogar donde va a ir el animal para garantizar que el entorno es seguro y adecuado. También se verifica que se disponga de todo lo necesario para recibir a la mascota, como comederos, juguetes, cama, etc.
\item \textbf{Contrato de adopción:} es indispensable para garantizar la protección legal de las mascotas adoptadas. El nuevo dueño debe comprometerse legalmente a proporcionar atención veterinaria y a no abandonar al animal, entre otros puntos. En la mayoría de las organizaciones, es habitual vacunar y colocar un chip al animal antes de que vaya al hogar del nuevo dueño.
\item \textbf{Recogida del animal:} el día de la recogida, se entrega todo el historial médico del animal (si se dispone de él) y la organización suele proporcionar consejos para los nuevos dueños.
\item \textbf{Seguimiento:} se requiere un seguimiento posterior a la adopción para garantizar el bienestar del animal y brindar apoyo o asesoramiento al dueño.
\end{itemize}

Todo este proceso es guiado por la organización correspondiente y es importante completarlo, aunque pueda resultar un poco tedioso. Para iniciar este proceso de adopción, la persona deberá ponerse en contacto previamente con las distintas protectoras o refugios en su área. Esta fase inicial puede presentar diferentes problemas a los adoptantes. En la mayoría de las protectoras, trabajan voluntarios y se sostienen de donaciones, por lo que es común que no cuenten con fondos para generar visibilidad. Muchas protectoras utilizan diversas redes sociales o sus propios sitios web para promocionarse, lo que implica que una persona interesada en adoptar podría tener que utilizar varias plataformas para ponerse en contacto con alguna. A veces, es difícil mantener una comunicación adecuada con diferentes protectoras incluso después de la adopción. Otro problema es que los futuros adoptantes pueden encontrar dificultades al elegir una mascota, ya que la información sobre las diferentes mascotas está dispersa y puede ser una tarea complicada buscar una que cumpla con todos los requisitos del adoptante.

Después de revisar los pasos y los posibles inconvenientes para solicitar un proceso de adopción, se propone una solución en forma de aplicación móvil. Con esta aplicación, se pretende solventar la problemática de la fase inicial de búsqueda de protectoras, proporcionando a los usuarios listas de protectoras cercanas y perros disponibles, incluyendo filtros para acotar la búsqueda. También se propone como una forma de facilitar a las protectoras la gestión de sus perros y posibles adoptantes. En definitiva, el objetivo principal de esta aplicación es centralizar la información de las organizaciones y sus mascotas, además de concienciar e incentivar a los usuarios a adoptar.


% Objetivos
\newpage
\subsection{Objetivos del proyecto}

Para satisfacer las necesidades del proyecto propuesto, se han definido una serie de objetivos divididos en dos categorías: los obligatorios, necesarios para garantizar el funcionamiento esperado de la aplicación, y los opcionales, que tienen como propósito añadir funcionalidades adicionales o facilitar su uso.


% Objetivo 1
\begin{table}[H]
	\captionsetup{width=0.95\linewidth}%
   	\captionsetup{singlelinecheck=false}%
	\captionsetup{list=no}%
	\captionsetup{font=bf}
	\caption{Objetivo 1}
	\begin{tabular}{ | m{3cm} | m{10cm} | }
		\hline \cellcolor{lightgray}\textbf{Título} & \cellcolor{gray} \textcolor{white}{\textit{Registro de usuarios con diferentes roles}}  \\ \hline
		\cellcolor{lightgray}\textbf{Tipo} & Obligatorio \\ \hline
		\cellcolor{lightgray}\textbf{Descripción} & La aplicación debe ofrecer la opción de registrarse como usuario o como protectora, los formularios correspondientes y almacenando los datos en la base de datos.  \\ \hline
	\end{tabular}
\end{table} 

% Objetivo 2
\begin{table}[H]
	\captionsetup{width=0.95\linewidth}%
   	\captionsetup{singlelinecheck=false}%
	\captionsetup{list=no}%
	\captionsetup{font=bf}
	\caption{Objetivo 2}
	\begin{tabular}{ | m{3cm} | m{10cm} | }
		\hline \cellcolor{lightgray}\textbf{Título} & \cellcolor{gray} \textcolor{white}{\textit{Registro de caninos}}  \\ \hline
		\cellcolor{lightgray}\textbf{Tipo} & Obligatorio \\ \hline
		\cellcolor{lightgray}\textbf{Descripción} & La aplicación debe ofrecer la posibilidad de registrar a varios caninos con diferentes características por de las protectoras y almacenar los datos en la base de datos.  \\ \hline
	\end{tabular}
\end{table} 

% Objetivo 3
\begin{table}[H]
	\captionsetup{width=0.95\linewidth}%
   	\captionsetup{singlelinecheck=false}%
	\captionsetup{list=no}%
	\captionsetup{font=bf}
	\caption{Objetivo 3}
	\begin{tabular}{ | m{3cm} | m{10cm} | }
		\hline \cellcolor{lightgray}\textbf{Título} & \cellcolor{gray} \textcolor{white}{\textit{Listado de caninos con filtros y búsqueda}}  \\ \hline
		\cellcolor{lightgray}\textbf{Tipo} & Obligatorio \\ \hline
		\cellcolor{lightgray}\textbf{Descripción} & La aplicación debe mostrar diferentes listas de caninos, permitiendo aplicar filtros rápidos y realizar búsquedas. \\ \hline
	\end{tabular}
\end{table} 

% Objetivo 4
\begin{table}[H]
	\captionsetup{width=0.95\linewidth}%
   	\captionsetup{singlelinecheck=false}%
	\captionsetup{list=no}%
	\captionsetup{font=bf}
	\caption{Objetivo 4}
	\begin{tabular}{ | m{3cm} | m{10cm} | }
		\hline \cellcolor{lightgray}\textbf{Título} & \cellcolor{gray} \textcolor{white}{\textit{Listado de usuarios con búsqueda}}  \\ \hline
		\cellcolor{lightgray}\textbf{Tipo} & Obligatorio \\ \hline
		\cellcolor{lightgray}\textbf{Descripción} & La aplicación debe mostrar diversas listas de usuarios que permitan realizar búsquedas en ellas.  \\ \hline
	\end{tabular}
\end{table} 

% Objetivo 5
\begin{table}[H]
	\captionsetup{width=0.95\linewidth}%
   	\captionsetup{singlelinecheck=false}%
	\captionsetup{list=no}%
	\captionsetup{font=bf}
	\caption{Objetivo 5}
	\begin{tabular}{ | m{3cm} | m{10cm} | }
		\hline \cellcolor{lightgray}\textbf{Título} & \cellcolor{gray} \textcolor{white}{\textit{Inclusión de sección de caninos favoritos}}  \\ \hline
		\cellcolor{lightgray}\textbf{Tipo} & Obligatorio \\ \hline
		\cellcolor{lightgray}\textbf{Descripción} & Se debe incluir un botón de favoritos en los perfiles de los caninos que permita a los usuarios añadir a su sección de favoritos a aquellos que les gusten.  \\ \hline
	\end{tabular}
\end{table} 

% Objetivo 6
\begin{table}[H]
	\captionsetup{width=0.95\linewidth}%
   	\captionsetup{singlelinecheck=false}%
	\captionsetup{list=no}%
	\captionsetup{font=bf}
	\caption{Objetivo 6}
	\begin{tabular}{ | m{3cm} | m{10cm} | }
		\hline \cellcolor{lightgray}\textbf{Título} & \cellcolor{gray} \textcolor{white}{\textit{Inclusión de sistema de mensajería entre usuarios}}  \\ \hline
		\cellcolor{lightgray}\textbf{Tipo} & Obligatorio \\ \hline
		\cellcolor{lightgray}\textbf{Descripción} & Se debe permitir a los usuarios abrir un chat con otros usuarios dentro de la aplicación. Además, los usuarios deberán recibir una notificación cuando reciban un mensaje.  \\ \hline
	\end{tabular}
\end{table} 

% Objetivo 7
\begin{table}[H]
	\captionsetup{width=0.95\linewidth}%
   	\captionsetup{singlelinecheck=false}%
	\captionsetup{list=no}%
	\captionsetup{font=bf}
	\caption{Objetivo 7}
	\begin{tabular}{ | m{3cm} | m{10cm} | }
		\hline \cellcolor{lightgray}\textbf{Título} & \cellcolor{gray} \textcolor{white}{\textit{Geolocalización y mapas con  resultados}}  \\ \hline
		\cellcolor{lightgray}\textbf{Tipo} & Obligatorio \\ \hline
		\cellcolor{lightgray}\textbf{Descripción} & Se debe incluir la ubicación para ordenar los resultados de la aplicación según la distancia del usuario. Además, se debe agregar una página con un mapa que muestre la ubicación de los resultados junto con una lista de los mismos.  \\ \hline
	\end{tabular}
\end{table} 

% Objetivo 8
\begin{table}[H]
	\captionsetup{width=0.95\linewidth}%
   	\captionsetup{singlelinecheck=false}%
	\captionsetup{list=no}%
	\captionsetup{font=bf}
	\caption{Objetivo 8}
	\begin{tabular}{ | m{3cm} | m{10cm} | }
		\hline \cellcolor{lightgray}\textbf{Título} & \cellcolor{gray} \textcolor{white}{\textit{Actualización de datos de caninos}}  \\ \hline
		\cellcolor{lightgray}\textbf{Tipo} & Obligatorio \\ \hline
		\cellcolor{lightgray}\textbf{Descripción} & Se debe permitir actualizar los atributos de los caninos, como el peso, edad, descripción, etc. Además, se debe proporcionar la opción de marcar si está disponible para adoptar/acoger o si ya ha sido adoptado. \\ \hline
	\end{tabular}
\end{table} 

% Objetivo 9
\begin{table}[H]
	\captionsetup{width=0.95\linewidth}%
   	\captionsetup{singlelinecheck=false}%
	\captionsetup{list=no}%
	\captionsetup{font=bf}
	\caption{Objetivo 9}
	\begin{tabular}{ | m{3cm} | m{10cm} | }
		\hline \cellcolor{lightgray}\textbf{Título} & \cellcolor{gray} \textcolor{white}{\textit{Actualización de datos de usuarios}}  \\ \hline
		\cellcolor{lightgray}\textbf{Tipo} & Obligatorio \\ \hline
		\cellcolor{lightgray}\textbf{Descripción} & Se debe permitir a un usuario actualizar sus datos personales y sus credenciales. \\ \hline
	\end{tabular}
\end{table} 

% Objetivo 10
\begin{table}[H]
	\captionsetup{width=0.95\linewidth}%
   	\captionsetup{singlelinecheck=false}%
	\captionsetup{list=no}%
	\captionsetup{font=bf}
	\caption{Objetivo 10}
	\begin{tabular}{ | m{3cm} | m{10cm} | }
		\hline \cellcolor{lightgray}\textbf{Título} & \cellcolor{gray} \textcolor{white}{\textit{Inclusión de barra de búsqueda de direcciones}}  \\ \hline
		\cellcolor{lightgray}\textbf{Tipo} & Opcional \\ \hline
		\cellcolor{lightgray}\textbf{Descripción} & Posibilidad de que un usuario pueda buscar su dirección y autocompletar automáticamente los diferentes campos del formulario. \\ \hline
	\end{tabular}
\end{table}

% Objetivo 11
\begin{table}[H]
	\captionsetup{width=0.95\linewidth}%
   	\captionsetup{singlelinecheck=false}%
	\captionsetup{list=no}%
	\captionsetup{font=bf}
	\caption{Objetivo 11}
	\begin{tabular}{ | m{3cm} | m{10cm} | }
		\hline \cellcolor{lightgray}\textbf{Título} & \cellcolor{gray} \textcolor{white}{\textit{Diseño de tema oscuro}}  \\ \hline
		\cellcolor{lightgray}\textbf{Tipo} & Opcional \\ \hline
		\cellcolor{lightgray}\textbf{Descripción} & Posibilidad de que un usuario pueda cambiar al tema oscuro dentro de la aplicación. \\ \hline
	\end{tabular}
\end{table}  

% Objetivo 12
\begin{table}[H]
	\captionsetup{width=0.95\linewidth}%
   	\captionsetup{singlelinecheck=false}%
	\captionsetup{list=no}%
	\captionsetup{font=bf}
	\caption{Objetivo 12}
	\begin{tabular}{ | m{3cm} | m{10cm} | }
		\hline \cellcolor{lightgray}\textbf{Título} & \cellcolor{gray} \textcolor{white}{\textit{Inclusión de un blog}}  \\ \hline
		\cellcolor{lightgray}\textbf{Tipo} & Opcional \\ \hline
		\cellcolor{lightgray}\textbf{Descripción} & Posibilidad de que un usuario pueda añadir publicaciones con imágenes y textos, además de añadir comentarios en las publicaciones para compartir ideas con otros usuarios. \\ \hline
	\end{tabular}
\end{table}  

% Objetivo 13
\begin{table}[H]
	\captionsetup{width=0.95\linewidth}%
   	\captionsetup{singlelinecheck=false}%
	\captionsetup{list=no}%
	\captionsetup{font=bf}
	\caption{Objetivo 13}
	\begin{tabular}{ | m{3cm} | m{10cm} | }
		\hline \cellcolor{lightgray}\textbf{Título} & \cellcolor{gray} \textcolor{white}{\textit{Compartición de caninos a través de un enlace}}  \\ \hline
		\cellcolor{lightgray}\textbf{Tipo} & Opcional \\ \hline
		\cellcolor{lightgray}\textbf{Descripción} & Posibilidad de incluir un botón de compartir que permita al usuario generar un enlace al can correspondiente para poder compartirlo a aplicaciones externas. \\ \hline
	\end{tabular}
\end{table}  

% Objetivo 14
\begin{table}[H]
	\captionsetup{width=0.95\linewidth}%
   	\captionsetup{singlelinecheck=false}%
	\captionsetup{list=no}%
	\captionsetup{font=bf}
	\caption{Objetivo 14}
	\begin{tabular}{ | m{3cm} | m{10cm} | }
		\hline \cellcolor{lightgray}\textbf{Título} & \cellcolor{gray} \textcolor{white}{\textit{Envío de imágenes dentro del chat}}  \\ \hline
		\cellcolor{lightgray}\textbf{Tipo} & Opcional \\ \hline
		\cellcolor{lightgray}\textbf{Descripción} & Posibilidad de enviar imágenes dentro del chat. \\ \hline
	\end{tabular}
\end{table} 

% Planificación
\newpage
\section{Gestión del proyecto}
\subsection{Planificación temporal}
En esta sección se definen todas las etapas en las que se dividirá el proyecto, los cuales se resumen en un diagrama de Gantt (ver Figura 1). Este diagrama abarca desde enero de 2024 hasta junio del mismo año, estableciendo un marco temporal para cada fase del proyecto y asignando recursos de manera eficiente para alcanzar los objetivos establecidos.

\begin{figure}[H]
	{\includegraphics[width=15cm]{diagram/GanntSmall2.png}\par}
	\caption{Diagrama de Gantt}
\end{figure}

Dentro del diagrama se pueden observar las diferentes etapas del proyecto:

\begin{itemize}[noitemsep]
	\item \textbf{Análisis:} esta etapa tiene una duración aproximada de dos semanas, durante la cual se definen todos los requisitos del sistema y se identifican los recursos humanos, de hardware y de software necesarios.
	\item \textbf{Estudio y aprendizaje:} se asigna un breve período previo para aprender a utilizar los lenguajes y servicios que se emplearán en el desarrollo de la aplicación.
	\item \textbf{Diseño:} se destinan casi tres semanas para definir temas, bocetos, la estructura de la base de datos y la arquitectura del sistema.
	\item \textbf{Desarrollo:} esta es la etapa más extensa, dividida en secciones para cada funcionalidad a desarrollar, a las cuales se les asigna un tiempo adecuado según los servicios y componentes involucrados que definirán su dificultad.
	\item \textbf{Pruebas y corrección de errores:} una vez completado el desarrollo, se realizan pruebas exhaustivas para identificar errores y faltas de funcionalidades. Se reserva tiempo suficiente para corregir errores o implementar mejoras de rendimiento si es necesario.
	\item \textbf{Documentación y memoria:} se dedica un período para recopilar todos los datos y elaborar la memoria en LaTeX, documentando el proceso y los resultados obtenidos en las etapas anteriores.
\end{itemize}

\newpage
\subsection{Estimación de costes}

Para el desarrollo del producto, se puede definir un presupuesto que abarque todos los recursos requeridos.

En este proyecto, participará únicamente una persona: la autora de la memoria. Se trata de una ingeniera de software, responsable de todos los aspectos del desarrollo. El horario laboral será de lunes a viernes, a media jornada. Para calcular el costo relacionado con el personal, se ha tomado como referencia el sueldo medio de un programador junior, que ronda los \textit{24.000€} anuales, lo que equivale a \textit{12.000€} a media jornada. Haciendo cálculos, al mes corresponden unos \textit{1.000€}. Dado que la duración del proyecto es de 5 meses y medio, el costo total en salarios para el empleado es de \textit{5.500€}.

Para este proyecto, el hardware disponible incluye un ordenador de sobremesa personal y una tablet Android.

\begin{itemize}[noitemsep]
	\item Ordenador de sobremesa, para el desarrollo y las pruebas, con los siguientes componentes:
		\begin{itemize}[noitemsep]
			\item \textbf{Procesador:} AMD Ryzen 7 3700X 8-Core Processor - 3.60 GHz
			\item \textbf{Memoria RAM:} DDR4 3000 2x16GB
			\item \textbf{Tarjeta gráfica:} NVIDIA GeForce GTX 1660 SUPER
			\item \textbf{Placa base:} ASUS TUF GAMING B550-PLUS WIFI II
			\item \textbf{Disipador:} Noctua NH-D15
		\end{itemize}
	\item Tablet Android Hi9plus, para realizar pruebas en un dispositivo Android físico.
\end{itemize}

Debido a que el equipo utilizado no es nuevo, se consideran los gastos del equipo teniendo en cuenta que un equipo informático se puede amortizar hasta 8 años. El precio aproximado del ordenador es de unos \textit{1.000€}. Un año tiene 12 meses, lo que resulta en 96 meses en total de amortización, lo que implica que cuesta unos \textit{11€} al mes. Dado que la duración del proyecto es de 5 meses y medio, el costo del ordenador sería de \textit{60,50€} en total. El precio de la tablet es de unos \textit{100€}, así que utilizando los mismos cálculos, el costo de la tablet es de \textit{5,72€} en total.

Con respecto al software, para el alcance de este proyecto y los objetivos que se pretenden alcanzar, al menos durante la primera fase, se ha optado por utilizar las versiones gratuitas del software utilizado, lo que resulta en un coste total de 0€.

\begin{itemize}[noitemsep]
	\item Firebase - Plan Spark - Guía de planes \cite{firebase_plans}
	\item Google APIs - Ofrecen créditos gratuitos de hasta \textit{200€} para cubir los gastos que se generen por el uso de la API. - Precios \cite{google_prices}
	\item IDE + Framework - Gratuitos
\end{itemize}


\begin{table}[H]
    \centering
    \begin{tabular}{ | m{5cm} | m{5cm} | m{5cm} | }
	    \hline \textbf{Tipo de recurso} & \textbf{Recurso} & \textbf{Coste (€)} \\ \hline
	    	Personal & Sueldo empleada & 5.500 \\ \hline
		Hardware & Ordenador & 60,50 \\ \hline
		Hardware & Tablet & 5,72 \\ \hline
	    	Software & Firebase & 0 \\ \hline
		Software & Google APIs & 0 \\ \hline
		Software & IDE + Framework & 0 \\ \hline
	   	- & -  & \textbf{Total: 5.566,22€} \\ \hline
    \end{tabular}
    \caption{Costes del proyecto}
    \label{tab:costes}
\end{table}


% Análisis
\newpage
\section{Análisis de requisitos}

% Definición y Especificación de Requisitos
\subsection{Definición y especificación de requisitos}

En esta sección se detallan los requisitos identificados para la aplicación que se desarrollará. Es fundamental definir y comprender estos requisitos para asegurar resultados óptimos en el desarrollo y diseño del sistema.

\subsubsection{Requisitos funcionales}

Los Requisitos Funcionales (RF) describen las acciones específicas que el sistema debe realizar, los servicios que debe proporcionar y cómo debe responder a diversas entradas. Estos requisitos se centran en el ``qué'' del sistema, delineando las funcionalidades clave que los usuarios esperan encontrar en la aplicación.

% RF1
\begin{table}[H]
\captionsetup{list=no}%
\captionsetup{justification=raggedright,singlelinecheck=false}
\caption{\textbf{RF1:} Registro de usuarios y protectoras.}
\label{tab:RF1}
	\begin{tabular}{|m{5cm}|m{10cm}|}
	\hline
	\textbf{Descripción} & Ofrecer un formulario de registro para usuarios y protectoras. \\ 
	\hline
	\textbf{Datos de Entrada} & Datos ingresados por el usuario en el formulario de registro. Incluyendo datos de contacto, correo electrónico y dirección. \\ 
	\hline
	\textbf{Datos de Salida} & Datos del usuario almacenados en la base de datos. \\ 
	\hline
\end{tabular}
\end{table}

% RF2
\begin{table}[H]
\captionsetup{list=no}%
\captionsetup{justification=raggedright,singlelinecheck=false}
\caption{\textbf{RF2:} Registro de caninos.}
\label{tab:RF2}
	\begin{tabular}{|m{5cm}|m{10cm}|}
	\hline
	\textbf{Descripción} & Ofrecer un formulario de registro para añadir perros con características específicas. \\ 
	\hline
	\textbf{Datos de Entrada} & Datos ingresados por la protectora, tales como raza, peso, edad, color y descripción. \\ 
	\hline
	\textbf{Datos de Salida} & Datos del perro almacenados en la base de datos, junto al id de su protectora. \\ 
	\hline
\end{tabular}
\end{table}

% RF3
\begin{table}[H]
\captionsetup{list=no}%
\captionsetup{justification=raggedright,singlelinecheck=false}
\caption{\textbf{RF3:} Mostrar listas de usuarios.}
\label{tab:RF23}
	\begin{tabular}{|m{5cm}|m{10cm}|}
	\hline
	\textbf{Descripción} & Mostrar diferentes listas de usuarios en la aplicación, con filtros predeterminados. \\ 
	\hline
	\textbf{Datos de Entrada} & Ninguno. \\ 
	\hline
	\textbf{Datos de Salida} & Listas de usuarios de la aplicación que cumplen con los filtros.  \\ 
	\hline
\end{tabular}
\end{table}

% RF4
\begin{table}[H]
\captionsetup{list=no}%
\captionsetup{justification=raggedright,singlelinecheck=false}
\caption{\textbf{RF4:} Mostrar listas de perros.}
\label{tab:RF4}
	\begin{tabular}{|m{5cm}|m{10cm}|}
	\hline
	\textbf{Descripción} & Mostrar diferentes listas de perros en la aplicación, con filtros predeterminados \\ 
	\hline
	\textbf{Datos de Entrada} & Ninguno. \\ 
	\hline
	\textbf{Datos de Salida} & Listas de perros de la aplicación que cumplen con los filtros. \\ 
	\hline
\end{tabular}
\end{table}

% RF5
\begin{table}[H]
\captionsetup{list=no}%
\captionsetup{justification=raggedright,singlelinecheck=false}
\caption{\textbf{RF5:} Capacidad de filtrado por propiedades en listas de perros.}
\label{tab:RF5}
	\begin{tabular}{|m{5cm}|m{10cm}|}
	\hline
	\textbf{Descripción} & Proporcionar diferentes filtros para las listas de perros, que corresponden con las diferentes propiedades que puede tener un perro en la aplicación. \\ 
	\hline
	\textbf{Datos de Entrada} & Valores de los filtros proporcionados por el usuario. \\ 
	\hline
	\textbf{Datos de Salida} & Lista de perros de la aplicación que cumplen con los filtros. \\ 
	\hline
\end{tabular}
\end{table}

% RF6
\begin{table}[H]
\captionsetup{list=no}%
\captionsetup{justification=raggedright,singlelinecheck=false}
\caption{\textbf{RF6:} Capacidad de filtrado por búsqueda en listas de perros.}
\label{tab:RF6}
	\begin{tabular}{|m{5cm}|m{10cm}|}
	\hline
	\textbf{Descripción} & Filtrar resultados de listas según el texto ingresado en una barra de búsqueda, la búsqueda se realiza en diferentes campos de los perros. \\ 
	\hline
	\textbf{Datos de Entrada} & Texto de búsqueda ingresado por el usuario. \\ 
	\hline
	\textbf{Datos de Salida} &  Lista de perros de la aplicación que cumplen con el criterio de búsqueda en alguno de los campos. \\ 
	\hline
\end{tabular}
\end{table}

% RF7
\begin{table}[H]
\captionsetup{list=no}%
\captionsetup{justification=raggedright,singlelinecheck=false}
\caption{\textbf{RF7:} Capacidad de filtrado por búsqueda en listas de usuarios..}
\label{tab:RF7}
	\begin{tabular}{|m{5cm}|m{10cm}|}
\hline
	\textbf{Descripción} & Filtrar resultados de listas según el texto ingresado en una barra de búsqueda, la búsqueda se realiza en diferentes campos de los usuarios. \\ 
	\hline
	\textbf{Datos de Entrada} & Texto de búsqueda ingresado por el usuario. \\ 
	\hline
	\textbf{Datos de Salida} &  Lista de usuarios de la aplicación que cumplen con el criterio de búsqueda en alguno de los campos. \\ 
	\hline
\end{tabular}
\end{table}

% RF8
\begin{table}[H]
\captionsetup{list=no}%
\captionsetup{justification=raggedright,singlelinecheck=false}
\caption{\textbf{RF8:}  Proporcionar botón de mapa de resultados en listas de usuarios.}
\label{tab:RF8}
	\begin{tabular}{|m{5cm}|m{10cm}|}
	\hline
	\textbf{Descripción} & Botón en listas de usuarios que redirige a una página con mapa con todos los resultados con ubicación. \\ 
	\hline
	\textbf{Datos de Entrada} & Ninguno \\ 
	\hline
	\textbf{Datos de Salida} & Botón con capacidad de redirección a la página del mapa. \\ 
	\hline
\end{tabular}
\end{table}

% RF9
\begin{table}[H]
\captionsetup{list=no}%
\captionsetup{justification=raggedright,singlelinecheck=false}
\caption{\textbf{RF9:} Proporcionar botón de favoritos en perfiles de perros.}
\label{tab:RF9}
	\begin{tabular}{|m{5cm}|m{10cm}|}
	\hline
	\textbf{Descripción} & Permitir a un usuario o protectora marcar como favorito a un perro. \\ 
	\hline
	\textbf{Datos de Entrada} & Click del usuario sobre el botón. \\ 
	\hline
	\textbf{Datos de Salida} & Datos del perro con la actualización de ids de usuarios que lo han marcado/desmarcado como favorito.\\ 
	\hline
\end{tabular}
\end{table}

% RF10
\begin{table}[H]
\captionsetup{list=no}%
\captionsetup{justification=raggedright,singlelinecheck=false}
\caption{\textbf{RF10:} Mostrar sección de perros favoritos. }
\label{tab:RF10}
	\begin{tabular}{|m{5cm}|m{10cm}|}
	\hline
	\textbf{Descripción} & Muestra una sección de perros marcados como favoritos en ese momento por el usuario en su página de inicio. \\ 
	\hline
	\textbf{Datos de Entrada} & Ninguno. \\ 
	\hline
	\textbf{Datos de Salida} & Lista horizontal de perros favoritos en la parte inferiro de la página de inicio de los usuarios. \\ 
	\hline
\end{tabular}
\end{table}

% RF11
\begin{table}[H]
\captionsetup{list=no}%
\captionsetup{justification=raggedright,singlelinecheck=false}
\caption{\textbf{RF11:} Mostrar adopciones recientes.}
\label{tab:RF11}
	\begin{tabular}{|m{5cm}|m{10cm}|}
	\hline
	\textbf{Descripción} & Muestra un swiper con imágenes de perros marcados como adoptados recientemente. \\ 
	\hline
	\textbf{Datos de Entrada} & Ninguno. \\ 
	\hline
	\textbf{Datos de Salida} & Swiper con imágenes y nombres de perros adoptados en la aplicación. \\ 
	\hline
\end{tabular}
\end{table}

% RF12
\begin{table}[H]
\captionsetup{list=no}%
\captionsetup{justification=raggedright,singlelinecheck=false}
\caption{\textbf{RF12:} Proporcionar botón de cerrar sesión.}
\label{tab:RF12}
	\begin{tabular}{|m{5cm}|m{10cm}|}
	\hline
	\textbf{Descripción} & Mostrar botón de cerrar sesión en el menú de acciones en la app bar de la aplicación. \\ 
	\hline
	\textbf{Datos de Entrada} & Ninguno. \\ 
	\hline
	\textbf{Datos de Salida} & Mostrar página de inicio de sesión, después de haber eliminado todos los datos relacionados con el usuario iniciado. \\ 
	\hline
\end{tabular}
\end{table}

% RF13
\begin{table}[H]
\captionsetup{list=no}%
\captionsetup{justification=raggedright,singlelinecheck=false}
\caption{\textbf{RF13:} Proporcionar botón de contacto.}
\label{tab:RF13}
	\begin{tabular}{|m{5cm}|m{10cm}|}
	\hline
	\textbf{Descripción} & Mostrar botón de contacto en el menú de acciones en la app bar de la aplicación. \\ 
	\hline
	\textbf{Datos de Entrada} & Ninguno. \\ 
	\hline
	\textbf{Datos de Salida} & Mostrar página de contacto. \\ 
	\hline
\end{tabular}
\end{table}

% RF14
\begin{table}[H]
\captionsetup{list=no}%
\captionsetup{justification=raggedright,singlelinecheck=false}
\caption{\textbf{RF14:} Botón de menu lateral en la app bar.}
\label{tab:RF14}
	\begin{tabular}{|m{5cm}|m{10cm}|}
	\hline
	\textbf{Descripción} & Mostrar botón que abre menú lateral. \\ 
	\hline
	\textbf{Datos de Entrada} & Ninguno. \\ 
	\hline
	\textbf{Datos de Salida} & Menu lateral con las páginas disponibles para ese usuario. \\ 
	\hline
\end{tabular}
\end{table}

% RF15
\begin{table}[H]
\captionsetup{list=no}%
\captionsetup{justification=raggedright,singlelinecheck=false}
\caption{\textbf{RF15:} Mostrar datos personales en el menú lateral.}
\label{tab:RF15}
	\begin{tabular}{|m{5cm}|m{10cm}|}
	\hline
	\textbf{Descripción} & Muestra los datos del usuario iniciado en el menú lateral. \\ 
	\hline
	\textbf{Datos de Entrada} & Ninguno. \\ 
	\hline
	\textbf{Datos de Salida} & Datos del usuario almacenados en la base de datos. \\ 
	\hline
\end{tabular}
\end{table}

% RF16
\begin{table}[H]
\captionsetup{list=no}%
\captionsetup{justification=raggedright,singlelinecheck=false}
\caption{\textbf{RF16:} Botón de 'Inicio' en el menú lateral.}
\label{tab:RF16}
	\begin{tabular}{|m{5cm}|m{10cm}|}
	\hline
	\textbf{Descripción} & Proporcionar botón para redirigir al usuario a la página de inicio de la aplicación. \\ 
	\hline
	\textbf{Datos de Entrada} &  Ninguno. \\ 
	\hline
	\textbf{Datos de Salida} &  Página de inicio de la aplicación. \\ 
	\hline
\end{tabular}
\end{table}

% RF17
\begin{table}[H]
\captionsetup{list=no}%
\captionsetup{justification=raggedright,singlelinecheck=false}
\caption{\textbf{RF17:} Botón de 'Perfil personal' en el menú lateral.}
\label{tab:RF17}
	\begin{tabular}{|m{5cm}|m{10cm}|}
	\hline
	\textbf{Descripción} & Proporcionar botón para redirigir al usuario a la página de inicio de la aplicación. \\ 
	\hline
	\textbf{Datos de Entrada} &  Ninguno. \\ 
	\hline
	\textbf{Datos de Salida} &  Página de perfil personal. \\ 
	\hline
\end{tabular}
\end{table}

% RF18
\begin{table}[H]
\captionsetup{list=no}%
\captionsetup{justification=raggedright,singlelinecheck=false}
\caption{\textbf{RF18:} Botón de 'Mensajes' en el menú lateral.}
\label{tab:RF18}
	\begin{tabular}{|m{5cm}|m{10cm}|}
	\hline
	\textbf{Descripción} & Proporcionar botón para redirigir al usuario a la página de chats de la aplicación. \\ 
	\hline
	\textbf{Datos de Entrada} &  Ninguno. \\ 
	\hline
	\textbf{Datos de Salida} &  Página de chats. \\ 
	\hline
\end{tabular}
\end{table}

% RF19
\begin{table}[H]
\captionsetup{list=no}%
\captionsetup{justification=raggedright,singlelinecheck=false}
\caption{\textbf{RF19:} Botón de 'Protectoras' en el menú lateral.}
\label{tab:RF19}
	\begin{tabular}{|m{5cm}|m{10cm}|}
	\hline
	\textbf{Descripción} & Proporcionar botón para redirigir al usuario a la lista de protectoras de la aplicación. \\ 
	\hline
	\textbf{Datos de Entrada} &  Ninguno. \\ 
	\hline
	\textbf{Datos de Salida} &  Lista de protectoras ya verificadas en la aplicación. \\ 
	\hline
\end{tabular}
\end{table}

% RF20
\begin{table}[H]
\captionsetup{list=no}%
\captionsetup{justification=raggedright,singlelinecheck=false}
\caption{\textbf{RF20:} Botón de 'Mis perros' en el menú lateral.}
\label{tab:RF20}
	\begin{tabular}{|m{5cm}|m{10cm}|}
	\hline
	\textbf{Descripción} & Proporcionar botón para redirigir al usuario a la lista de sus perros de la aplicación \\ 
	\hline
	\textbf{Datos de Entrada} &  Ninguno. \\ 
	\hline
	\textbf{Datos de Salida} &  Lista de perros que ha dado de alta el usuario en la aplicación. \\ 
	\hline
\end{tabular}
\end{table}

% RF21
\begin{table}[H]
\captionsetup{list=no}%
\captionsetup{justification=raggedright,singlelinecheck=false}
\caption{\textbf{RF21:} Botón de 'Usuarios' en el menú lateral.}
\label{tab:RF21}
	\begin{tabular}{|m{5cm}|m{10cm}|}
	\hline
	\textbf{Descripción} & Proporcionar botón para redirigir al usuario a la lista de usuarios de la aplicación. \\ 
	\hline
	\textbf{Datos de Entrada} &  Ninguno. \\ 
	\hline
	\textbf{Datos de Salida} &  Lista de usuarios que no son protectoras en la aplicación. \\ 
	\hline
\end{tabular}
\end{table}

% RF22
\begin{table}[H]
\captionsetup{list=no}%
\captionsetup{justification=raggedright,singlelinecheck=false}
\caption{\textbf{RF22:} Botón de 'Información legal' en el menú lateral.}
\label{tab:RF22}
	\begin{tabular}{|m{5cm}|m{10cm}|}
	\hline
	\textbf{Descripción} & Proporcionar botón para redirigir al usuario a la página de información legal de la aplicación. \\ 
	\hline
	\textbf{Datos de Entrada} &  Ninguno. \\ 
	\hline
	\textbf{Datos de Salida} &  Página de información legal de la aplicación. \\ 
	\hline
\end{tabular}
\end{table}

% RF23
\begin{table}[H]
\captionsetup{list=no}%
\captionsetup{justification=raggedright,singlelinecheck=false}
\caption{\textbf{RF23:} Edición de datos personales y de contacto.}
\label{tab:RF23}
	\begin{tabular}{|m{5cm}|m{10cm}|}
	\hline
	\textbf{Descripción} & Proporcionar formularios de edición para los datos del usuario. \\ 
	\hline
	\textbf{Datos de Entrada} & Datos ingresados por el usuario en el formulario. Puede haber datos sin actualizar. \\ 
	\hline
	\textbf{Datos de Salida} &  Datos del usuario actualizados en la base de datos. \\ 
	\hline
\end{tabular}
\end{table}

% RF24
\begin{table}[H]
\captionsetup{list=no}%
\captionsetup{justification=raggedright,singlelinecheck=false}
\caption{\textbf{RF24:} Edición de datos de perros.}
\label{tab:RF24}
	\begin{tabular}{|m{5cm}|m{10cm}|}
	\hline
	\textbf{Descripción} & Propocionar formularios de edición para los datos de los caninos. \\ 
	\hline
	\textbf{Datos de Entrada} &  Datos ingresados por el usuario en el formulario. Puede haber datos sin actualizar.  \\ 
	\hline
	\textbf{Datos de Salida} &   Datos del canino actualizados en la base de datos. \\ 
	\hline
\end{tabular}
\end{table}

% RF25
\begin{table}[H]
\captionsetup{list=no}%
\captionsetup{justification=raggedright,singlelinecheck=false}
\caption{\textbf{RF25:} Recuperación de contraseña.}
\label{tab:RF25}
	\begin{tabular}{|m{5cm}|m{10cm}|}
	\hline
	\textbf{Descripción} & Proporcionar formularios de recuperación de contraseña para los usuarios y protectoras. \\ 
	\hline
	\textbf{Datos de Entrada} & Correo electrónico ingresado por el usuario. \\ 
	\hline
	\textbf{Datos de Salida} & Correo electrónico con las instrucciones de recuperación de contraseña. \\ 
	\hline
\end{tabular}
\end{table}

% RF26
\begin{table}[H]
\captionsetup{list=no}%
\captionsetup{justification=raggedright,singlelinecheck=false}
\caption{\textbf{RF26:} Mostrar página de perfil de usuario/protectora.}
\label{tab:RF26}
	\begin{tabular}{|m{5cm}|m{10cm}|}
	\hline
	\textbf{Descripción} & Proporcionar una interfaz que condense toda la información de un usuario en una sola página. \\ 
	\hline
	\textbf{Datos de Entrada} & Ninguno. \\ 
	\hline
	\textbf{Datos de Salida} & Datos del usuario correspondinte, incluyendo foto de perfil, correo y los perros (si los tiene). Además incluirá el botón de contacto para abir chat. \\ 
	\hline
\end{tabular}
\end{table}

% RF27
\begin{table}[H]
\captionsetup{list=no}%
\captionsetup{justification=raggedright,singlelinecheck=false}
\caption{\textbf{RF27:} Mostrar página de perfil de canino.}
\label{tab:RF27}
	\begin{tabular}{|m{5cm}|m{10cm}|}
	\hline
	\textbf{Descripción} & Proporcionar una interfaz que condense toda la información relacionada con un perro en una sola página. \\ 
	\hline
	\textbf{Datos de Entrada} & Ninguno. \\ 
	\hline
	\textbf{Datos de Salida} & Datos del perro correspondiente, incluyendo foto de perfil, características del canino (edad, raza, peso, descripción...). Además de un mapa indicando su ubicación, el bóton de favoritos y de compartir. También incluye el botón de abrir chat. Contendrá los botones de edición si el perro es del usuario que visita el perfil. \\ 
	\hline
\end{tabular}
\end{table}

% RF28
\begin{table}[H]
\captionsetup{list=no}%
\captionsetup{justification=raggedright,singlelinecheck=false}
\caption{\textbf{RF28:} Compartir perros a través de enlace.}
\label{tab:RF28}
	\begin{tabular}{|m{5cm}|m{10cm}|}
	\hline
	\textbf{Descripción} & Proporcionar un botón que genere un enlace para redirigir a un usuario al perfil de ese perro en concreto. \\ 
	\hline
	\textbf{Datos de Entrada} & Ninguno. \\ 
	\hline
	\textbf{Datos de Salida} & Url a la aplicación con los parámetros correspondientes para redirigir al perfil de perro. \\ 
	\hline
\end{tabular}
\end{table}

% RF29
\begin{table}[H]
\captionsetup{list=no}%
\captionsetup{justification=raggedright,singlelinecheck=false}
\caption{\textbf{RF29:} Página del mapa.}
\label{tab:RF29}
	\begin{tabular}{|m{5cm}|m{10cm}|}
	\hline
	\textbf{Descripción} & Proporcionar una interfaz que contenga un mapa con diferentes marcadores para cada una de las protectoras y un listado de las mismas debajo del mapa. \\ 
	\hline
	\textbf{Datos de Entrada} & Ninguno. \\ 
	\hline
	\textbf{Datos de Salida} & Mapa con marcadores y listado de protectoras, con elementos que pueden mover la cámara a los diferentes marcadores del mapa. \\ 
	\hline
\end{tabular}
\end{table}


% RF30
\begin{table}[H]
\captionsetup{list=no}%
\captionsetup{justification=raggedright,singlelinecheck=false}
\caption{\textbf{RF30:} Página de chats.}
\label{tab:RF30}
	\begin{tabular}{|m{5cm}|m{10cm}|}
	\hline
	\textbf{Descripción} & Proporcionar una interfaz que contenga un listado de chats abiertos dentro de la aplicación. \\ 
	\hline
	\textbf{Datos de Entrada} & Ninguno. \\ 
	\hline
	\textbf{Datos de Salida} & Listado de chats disponibles en la aplicación, incluyen una preview del último mensaje y datos del usuario correspondiente. \\ 
	\hline
\end{tabular}
\end{table}


% RF31
\begin{table}[H]
\captionsetup{list=no}%
\captionsetup{justification=raggedright,singlelinecheck=false}
\caption{\textbf{RF31:} Enviar mensajes al chat.}
\label{tab:RF31}
	\begin{tabular}{|m{5cm}|m{10cm}|}
	\hline
	\textbf{Descripción} & Permitir mandar mensajes a otro usuario dentro de la aplicación. \\ 
	\hline
	\textbf{Datos de Entrada} & Texto ingresado por el usuario. \\ 
	\hline
	\textbf{Datos de Salida} & Mensaje guardado en la base de datos Se ctualizar la tabla de chats de la base de datos con el último mensaje enviado y generar el id del chat si es el primer mensaje. Notificación al usuario receptor. \\ 
	\hline
\end{tabular}
\end{table}


% RF32
\begin{table}[H]
\captionsetup{list=no}%
\captionsetup{justification=raggedright,singlelinecheck=false}
\caption{\textbf{RF32:} Enviar imágenes al chat.}
\label{tab:RF32}
	\begin{tabular}{|m{5cm}|m{10cm}|}
\hline
	\textbf{Descripción} & Permitir mandar mensajes a otro usuario dentro de la aplicación. \\ 
	\hline
	\textbf{Datos de Entrada} & Imagen adjuntada por el usuario. \\ 
	\hline
	\textbf{Datos de Salida} & Imagen guardada además de su referencia en la tabla de mensaje. Se la tabla de chats de la base de datos con el último mensaje enviado y generar el id del chat si es el primer mensaje. Notificación al usuario receptor. \\ 
	\hline
\end{tabular}
\end{table}


% RF33
\begin{table}[H]
\captionsetup{list=no}%
\captionsetup{justification=raggedright,singlelinecheck=false}
\caption{\textbf{RF33:} Abrir un chat nuevo.}
\label{tab:RF33}
	\begin{tabular}{|m{5cm}|m{10cm}|}
	\hline
	\textbf{Descripción} & Proporcionar un botón de contacto para abrir un chat con algún usuario dentro de la aplicación. \\ 
	\hline
	\textbf{Datos de Entrada} & Ninguno. \\ 
	\hline
	\textbf{Datos de Salida} & Interfaz del chat, con la cabecera que incluye los datos del usuario receptor. \\ 
	\hline
\end{tabular}
\end{table}

% RF34
\begin{table}[H]
\captionsetup{list=no}%
\captionsetup{justification=raggedright,singlelinecheck=false}
\caption{\textbf{RF34:} Borrar un chat.}
\label{tab:RF34}
	\begin{tabular}{|m{5cm}|m{10cm}|}
	\hline
	\textbf{Descripción} & Deslizar un chat para eliminarlo de la lista de chats disponibles \\ 
	\hline
	\textbf{Datos de Entrada} & Ninguno. \\ 
	\hline
	\textbf{Datos de Salida} & Lista actualizada sin el chat que se acaba de eliminar \\ 
	\hline
\end{tabular}
\end{table}

% RF35
\begin{table}[H]
\captionsetup{list=no}%
\captionsetup{justification=raggedright,singlelinecheck=false}
\caption{\textbf{RF35:} Marcar/desmarcar perro como adoptado.}
\label{tab:RF35}
	\begin{tabular}{|m{5cm}|m{10cm}|}
	\hline
	\textbf{Descripción} & Proporcionar un botón dentro del perfil del perro que permite a una protectora marcar/desmarcar a un perro de la aplicación como adoptado. \\ 
	\hline
	\textbf{Datos de Entrada} & Estado nuevo según si se quiere marcar o desmarcar. \\ 
	\hline
	\textbf{Datos de Salida} & Perfil del canino actualizado y una etiqueta que indica si está adoptado. \\ 
	\hline
\end{tabular}
\end{table}


% RF36
\begin{table}[H]
\captionsetup{list=no}%
\captionsetup{justification=raggedright,singlelinecheck=false}
\caption{\textbf{RF36:} Página de contacto.}
\label{tab:RF36}
	\begin{tabular}{|m{5cm}|m{10cm}|}
	\hline
	\textbf{Descripción} & Proporcionar una interfaz que rehúna datos de contacto de administradores. \\ 
	\hline
	\textbf{Datos de Entrada} & Ninguno. \\ 
	\hline
	\textbf{Datos de Salida} & Interfaz que incluye correo electrónico y un botón de contacto para abrir un chat con un administrador. \\ 
	\hline
\end{tabular}
\end{table}

% RF37
\begin{table}[H]
\captionsetup{list=no}%
\captionsetup{justification=raggedright,singlelinecheck=false}
\caption{\textbf{RF37:} Página de información legal.}
\label{tab:RF37}
	\begin{tabular}{|m{5cm}|m{10cm}|}
	\hline
	\textbf{Descripción} & Proporcionar una interfaz que rehúna toda la información legal y advertencias. \\ 
	\hline
	\textbf{Datos de Entrada} & Ninguno. \\ 
	\hline
	\textbf{Datos de Salida} & Interfaz que incluye toda la información legal de la aplicación. \\ 
	\hline
\end{tabular}
\end{table}

% RF38
\begin{table}[H]
\captionsetup{list=no}%
\captionsetup{justification=raggedright,singlelinecheck=false}
\caption{\textbf{RF38:} Proporcionar página de inició según rol.}
\label{tab:RF38}
	\begin{tabular}{|m{5cm}|m{10cm}|}
	\hline
	\textbf{Descripción} & Proporcionar una interfaz única según el rol después de iniciar sesión. \\ 
	\hline
	\textbf{Datos de Entrada} & Correo electrónico y contraseña para iniciar sesión. \\ 
	\hline
	\textbf{Datos de Salida} & Página de inicio que varía según el rol. \\ 
	\hline
\end{tabular}
\end{table}

% RF39
\begin{table}[H]
\captionsetup{list=no}%
\captionsetup{justification=raggedright,singlelinecheck=false}
\caption{\textbf{RF39:} Página de administrador.}
\label{tab:RF39}
	\begin{tabular}{|m{5cm}|m{10cm}|}
	\hline
	\textbf{Descripción} & Proporcionar una página para los administradores. \\ 
	\hline
	\textbf{Datos de Entrada} & Correo electrónico y contraseña de administrador para iniciar sesión. \\ 
	\hline
	\textbf{Datos de Salida} & Interfaz para los administradores, que incluye listas con diferentes filtros de usuarios y caninos para manejar usuarios y caninos. \\ 
	\hline
\end{tabular}
\end{table}

% RF40
\begin{table}[H]
\captionsetup{list=no}%
\captionsetup{justification=raggedright,singlelinecheck=false}
\caption{\textbf{RF40:} Búsqueda de direcciones.}
\label{tab:RF40}
	\begin{tabular}{|m{5cm}|m{10cm}|}
	\hline
	\textbf{Descripción} & Proporcionar una barra de búsqueda de direcciones para facilitar la introducción de datos. \\ 
	\hline
	\textbf{Datos de Entrada} & Una cadena de caracteres que puede contener el nombre de la calle, la ciudad, el código postal... \\ 
	\hline
	\textbf{Datos de Salida} & Un listado de direcciones que coinciden con la búsqueda, al seleccionar una, se autocompletan los datos del formulario. \\ 
	\hline
\end{tabular}
\end{table}

% RF41
\begin{table}[H]
\captionsetup{list=no}%
\captionsetup{justification=raggedright,singlelinecheck=false}
\caption{\textbf{RF41:} Cambiar a tema oscuro.}
\label{tab:RF41}
	\begin{tabular}{|m{5cm}|m{10cm}|}
	\hline
	\textbf{Descripción} & Proporcionar una barra de búsqueda de direcciones para facilitar la introducción de datos. \\ 
	\hline
	\textbf{Datos de Entrada} & Ninguno.\\ 
	\hline
	\textbf{Datos de Salida} & Interfaz con el tema oscuro.  \\ 
	\hline
\end{tabular}
\end{table}

% RF42
\begin{table}[H]
\captionsetup{list=no}%
\captionsetup{justification=raggedright,singlelinecheck=false}
\caption{\textbf{RF42:} Subir post al blog.}
\label{tab:RF42}
	\begin{tabular}{|m{5cm}|m{10cm}|}
	\hline
	\textbf{Descripción} & Permitir a un usuario subir un post al blog. \\ 
	\hline
	\textbf{Datos de Entrada} & Contenido que puede ser texto o imágenes. \\ 
	\hline
	\textbf{Datos de Salida} & Post publicado en el blog de la aplicación. \\ 
	\hline
\end{tabular}
\end{table}

% RF43
\begin{table}[H]
\captionsetup{list=no}%
\captionsetup{justification=raggedright,singlelinecheck=false}
\caption{\textbf{RF43:} Comentar post del blog.}
\label{tab:RF43}
	\begin{tabular}{|m{5cm}|m{10cm}|}
	\hline
	\textbf{Descripción} & Permitir a un usuario comentar un post en el blog. \\ 
	\hline
	\textbf{Datos de Entrada} & Texto que puede incluir emoticonos. \\ 
	\hline
	\textbf{Datos de Salida} & Comentario publicado en el post. \\ 
	\hline
\end{tabular}
\end{table}


\subsubsection{Requisitos no funcionales}

Los Requisitos No Funcionales (RNF) son responsables de proporcionar al usuario una experiencia robusta y óptima a lo largo de la aplicación. Para el desarrollo de la aplicación se han considerado los siguientes puntos:

\begin{itemize}[noitemsep]
\item \textbf{Rendimiento:} La aplicación debe responder rápidamente a las interacciones de los usuarios y ejecutar las operaciones en un breve período de tiempo.
	\begin{itemize}[noitemsep]
	\item Se propone un tiempo máximo de 3 segundos como tiempo óptimo para cargar una página completa.
	\item El tiempo de respuesta del servidor debe ser menor de 2 segundos.
	\item La aplicación debe manejar múltiples peticiones sin una degradación perceptible del rendimiento, con un límite de 10.000 solicitudes.
	\item La base de datos debe realizar consultas simples y complejas en menos de un segundo.
	\item En el caso de búsquedas y filtrados, el tiempo de procesamiento puede ser mayor de 3 segundos, pero menor de 6.
	\item La aplicación debe manejar alrededor de 10.000 usuarios concurrentes sin sufrir una degradación del rendimiento.
	\end{itemize}
\item \textbf{Escalabilidad:} El código debe ser mantenible, permitiendo añadir o quitar funcionalidades sin comprometer el comportamiento de la aplicación.
	\begin{itemize}[noitemsep]
	\item Los componentes de la aplicación deben estar claramente definidos y documentados.
	\item Un componente base debe recoger todas las propiedades comunes, por ejemplo, un componente base para los botones debe incluir el tamaño de la fuente, la forma del botón, colores, etc.
	\item Las funcionalidades desarrolladas deben ser fácilmente deprecables si es necesario, sin dependencias innecesarias entre ellas.
	\item Los nuevos componentes no deben afectar el funcionamiento de la aplicación.
	\item La actualización de un componente base debe reflejarse en todos los componentes de ese tipo.
	\end{itemize}
\item \textbf{Usabilidad:} La aplicación debe ser intuitiva y atractiva, permitiendo su uso sencillo incluso por usuarios poco experimentados.
	\begin{itemize}[noitemsep]
	\item La interfaz debe ser limpia y organizada.
	\item Los colores de la aplicación deben ser coherentes.
	\item La navegación debe cumplir la norma de los tres clics: cualquier usuario debe alcanzar la información crítica en un máximo de tres clics.
	\item La aplicación debe ser intuitiva o proporcionar guías de uso si es necesario.
	\item Se debe testear la aplicación con usuarios reales antes del primer despliegue para realizar posibles correcciones.
	\end{itemize}
\item \textbf{Confiabilidad:} La aplicación debe manejar los errores de manera adecuada.
	\begin{itemize}[noitemsep]
	\item Los usuarios deben ser informados con mensajes apropiados si ocurre un error crítico.
	\item Durante el desarrollo, se deben añadir mecanismos para minimizar los errores conocidos.
	\item Las copias de seguridad son indispensables para garantizar que no se pierdan datos ante un error crítico.
	\item Se deben implementar mecanismos de manejo de errores en todas las operaciones críticas (lectura y escritura) para evitar errores graves.
	\end{itemize}
\item \textbf{Seguridad:} La aplicación debe proteger los datos de los usuarios y prevenir accesos no autorizados, asegurando que la información sea real y fehaciente.
	\begin{itemize}[noitemsep]
	\item Las contraseñas y datos sensibles deben estar cifrados.
	\item La aplicación no debe permitir acceso a usuarios sin cuenta.
	\item No se permitirá dar de alta a perros en protectoras no verificadas por administradores, para evitar fraudes.
	\item Un administrador debe tener acceso a listas para verificar información y eliminar datos si es necesario.
	\item Deben implementarse mecanismos de prevención contra ataques comunes como inyecciones SQL, XSS y CSRF.
	\end{itemize}
\item \textbf{Documentación:} Se debe almacenar toda la información relevante a lo largo de todas las etapas, ya sea en el código, en documentos aparte o en la memoria del proyecto.
\end{itemize}

\subsection{Diagramas de casos de uso}

Un caso de uso es una descripción detallada de cómo los usuarios interactúan con un sistema para lograr un objetivo específico. Estos pueden ser representados gráficamente mediante diagramas que muestran la interacción entre el usuario y el sistema para alcanzar dicho objetivo. A continuación se incluyen los diagramas de casos de uso definidos para cada uno de los roles: usuario, protectora y administrador.

%Caso de uso: Usuario
\begin{figure}[H]
	\begin{center}
		{\includegraphics[width=5cm]{diagram/UserC.png}\par}
		\caption{Diagrama de casos de uso: usuario.}
	\end{center}
\end{figure}


%Caso de uso: Protectora
\begin{figure}[H]
	\begin{center}
		{\includegraphics[width=5cm]{diagram/Shelter.png}\par}
		\caption{Diagrama de casos de uso: protectora.}
	\end{center}
\end{figure}


%Caso de uso: Administrador
\begin{figure}[H]
	\begin{center}
		{\includegraphics[width=5cm]{diagram/Admin.png}\par}
		\caption{Diagrama de casos de uso: administrador.}
	\end{center}
\end{figure}





% Recursos
\newpage
\section{Herramientas y software utilizado}

En esta sección se definen las herramientas y librerías necesarias para el desarrollo de la aplicación.

El sistema operativo utilizado es el que estaba instalado previamente en el ordenador personal.

A la hora de seleccionar el framework para utilizar durante el desarrollo, se parte de que se necesita crear una aplicación Android. Existen diferentes frameworks disponibles, como \textit{React Native}, \textit{Flutter}, \textit{Kotlin Multiplatform}, etc., que ofrecen integración con Android para facilitar el desarrollo. En este proyecto, se ha optado por utilizar Flutter, dado que se ha trabajado previamente con él. Este framework, además, permite realizar un desarrollo multiplataforma y ofrece un alto rendimiento.

El lenguaje utilizado ha sido \textit{Dart}. Este lenguaje es muy similar a Java y está orientado a objetos. Se utiliza principalmente para el desarrollo de aplicaciones del lado del cliente. Es un lenguaje bastante popularizado. Uno de los motivos para optar por este lenguaje ha sido din duda el mantenimiento que recibe a día de hoy y su simplicidad para programar. Para añadir funcionalidades extra, se han escogido varias bibliotecas que ofrecen microservicios o widgets.

Por último, para la base de datos, se ha optado por utilizar \textit{Firebase}. Se trata de una plataforma de desarrollo de aplicaciones web muy popularizada, también desarrollada por Google. Ofrece diversos servicios para la autenticación, el hosting de la aplicación y bases de datos en tiempo real. Esto lo convierte en el candidato adecuado para manejar funciones como el chat, la funcionalidad de compartir y otras características que nuestra aplicación requiere.

\begin{itemize}[noitemsep]
	\item \textbf{Sistema Operativo:} Windows 11 x64
	\item \textbf{IDE:} Android Studio Hedgehog | 2023.1.1 Patch 2
	\item \textbf{Framework:} Flutter 3.19.0 -  \href{https://flutter.dev/6}{Flutter DEV} \cite{flutter_dev}
	\item \textbf{Lenguaje:} Dart 3.3.0 - \href{https://dart.dev/}{Dart DEV} \cite{dart_dev}
	\item \textbf{Paquetes:}
		\begin{itemize}[noitemsep]
		  \item \texttt{cupertino\_icons: \^{}1.0.2}
		  \item \texttt{firebase\_core: \^{}2.24.2}
		  \item \texttt{firebase\_database: \^{}10.3.8}
		  \item \texttt{firebase\_storage:}
		  \item \texttt{cloud\_firestore: \^{}4.13.6}
		  \item \texttt{firebase\_auth: \^{}4.15.3}
		  \item \texttt{responsive\_sizer: \^{}3.3.0+1}
		  \item \texttt{flutter\_login: \^{}5.0.0}
		  \item \texttt{google\_fonts: \^{}4.0.4}
		  \item \texttt{resize: \^{}1.0.0}
		  \item \texttt{awesome\_notifications: \^{}0.9.2}
		  \item \texttt{csc\_picker: \^{}0.2.7}
		  \item \texttt{map\_address\_picker: \^{}0.3.5}
		  \item \texttt{geocoding: \^{}3.0.0}
		  \item \texttt{search\_map\_location: 0.0.6}
		  \item \texttt{image\_picker: \^{}1.0.7}
		  \item \texttt{image\_cropper: \^{}4.0.1}
		  \item \texttt{path\_provider: \^{}2.1.2}
		  \item \texttt{flutter\_image\_compress: \^{}2.1.0}
		  \item \texttt{address\_form: \^{}0.0.2}
		  \item \texttt{address: \^{}0.1.0+2}
		  \item \texttt{google\_maps\_webservice: \^{}0.0.20-nullsafety.5}
		  \item \texttt{meta\_validator: \^{}0.0.2}
		  \item \texttt{geolocator: \^{}9.0.2}
		  \item \texttt{dropdown\_search: \^{}5.0.6}
		  \item \texttt{card\_swiper: \^{}3.0.1}
		  \item \texttt{filter\_list: \^{}1.0.2}
		  \item \texttt{flutter\_filter\_dialog: \^{}1.2.0}
		  \item \texttt{choice: \^{}2.3.2}
		  \item \texttt{animate\_gradient: \^{}0.0.2+1}
		  \item \texttt{firebase\_messaging: \^{}14.9.1}
		  \item \texttt{flutter\_chat\_bubble: \^{}2.0.2}
		  \item \texttt{chat\_bubbles: \^{}1.6.0}
		  \item \texttt{share\_plus: \^{}9.0.0}
		  \item \texttt{app\_links: \^{}6.0.1}
		  \item \texttt{url\_launcher: \^{}6.2.6}
		  \item \texttt{firebase\_dynamic\_links: \^{}5.5.4}
		  \item \texttt{go\_router: \^{}14.0.2}
		  \item \texttt{get: \^{}4.6.6}
		  \item \texttt{linkwell: \^{}2.0.6}
		  \item \texttt{toastification: \^{}1.0.0}
		\end{itemize}
	\item \textbf{Base de datos:} Firebase database - Firebase console \cite{firebase_console}
\end{itemize}

% Diseño
\newpage
\section{Diseño de la Aplicación}

En esta parte se incluye toda la estructura de la aplicación, es decir, tanto la interfaz de usuario como el backend y la base de datos.

\subsection{La base de datos}

Se deben almacenar todas las entidades y sus atributos, definiendo claramente el tipo de cada una.

\subsubsection{Guardado de usuarios}

Los usuarios son otra entidad de gran relevancia dentro de la aplicación. Aunque un usuario tiene bastantes datos dentro de la aplicación, muchos de ellos se almacenan en otras tablas y se accede a ellos referenciando sus IDs. Además, para manejar los registros de usuarios se utiliza el servicio de autenticación que proporciona Firebase, el cual facilita las herramientas para registrar usuarios y permite almacenar tanto el correo electrónico como la contraseña, entre otras cosas. En esta tabla se guarda toda la información de los usuarios, independientemente de su rol, que puede ser: \textbf{user}, \textbf{company} o \textbf{admin}, indicando si el usuario es un usuario estándar, una protectora o un administrador, respectivamente.

% usuarios
\begin{table}[H]
\captionsetup{justification=raggedright,singlelinecheck=false}
\caption{\textbf{Tabla: usuarios}}
\label{tab:requests}
	\begin{tabular}{|m{3.2cm}|m{2cm}|m{5cm}|m{3cm}|}
	\hline
	\textbf{Campo} & \textbf{Tipo} & \textbf{Dato} & \textbf{Obligatorio} \\ 
	\hline
	\textbf{chats} & array & IDs de los chats que pertenecen al usuario &  Sí \\ 
	\hline
	\textbf{email} & string &  ID del chat desde el que se manda la notificación &  Sí \\ 
	\hline
	\textbf{profilePic} & string & URL de la imagen de foto de perfil & Sí \\ 
	\hline
	\textbf{rol} & string & Rol del usuario & Sí \\ 
	\hline
	\textbf{username} & string & Nombre de usuario & Sí \\ 
	\hline
	\textbf{verified} & string & Indica si ha sido verificado por un administrador & Sí, solo si su rol es protectora \\ 
	\hline
	\end{tabular}
\end{table}



A continuación, se muestra un ejemplo de usuario guardado en la base de datos:

\begin{verbatim}
{
  "chats": [
    "wfmRPtHiGvUI9vi9Fs7k"
  ],
  "email": "protectora@test.com",
  "profilePic": "",
  "rol": "company",
  "username": "Protectora super chula",
  "verified": true
}
\end{verbatim}

\subsubsection{Guardado de caninos}

Esta entidad es una de las más importantes de la aplicación, ya que es la principal afectada por la mayoría de funcionalidades que se proponen. Principalmente, esta tabla contendrá todas las propiedades de los caninos que se manejarán dentro de la app, aunque hay algunas consideraciones importantes para algunas de ellas. Por ejemplo, para el género se han propuesto dos valores fijos que luego se transformarán en la cadena correspondiente cuando se carguen en la interfaz de usuario. Los valores son \textbf{male} y \textbf{female}. Se propone esta solución para cuando se decida a futuro añadir traducciones en la aplicación. Por otro lado, para almacenar la edad, se ha optado por hacerlo junto a las unidades, ya que así se le da al usuario la posibilidad de especificar mejor la edad del can (el caso de uso es muy concreto, pero añade valor). Los valores de las unidades son fijos, pudiendo ser:

\begin{itemize}[noitemsep]
	\item \textbf{Meses}
	\item \textbf{Años}
\end{itemize}

Lo mismo ocurre con el peso se opta por almacenarlo junto a las unidades. Independientemente de las unidades que marque el usuario, el peso en la base de datos siempre será tratado como Kg para facilitar el filtrado. Cuando se cargue en la interfaz de usuario, se mostrarán las unidades y el valor original, que pueden ser:

\begin{itemize}[noitemsep]
	\item \textbf{Gramos}
	\item \textbf{Kg}
\end{itemize}

Una alternativa sería almacenar los caninos favoritos en la tabla de usuarios. Sin embargo, almacenar los usuarios que han marcado al canino como favorito en esta tabla facilita las actualizaciones en tiempo real de los resultados. Se realizaron algunas pruebas con la primera alternativa que requería que para mostrar el número de veces que un canino estaba marcado como favorito, hubiera que consultar todas las entradas existentes de usuarios. Por otra parte, requería recargar constantemente los resultados de los caninos en las listas tras marcar o desmarcar un favorito, lo que reducía considerablemente el rendimiento de la aplicación. Con este modelo, se puede trabajar únicamente escuchando los cambios en las diferentes entradas de la tabla para los caninos y actualizar la información en pantalla en tiempo real sin comprometer el rendimiento. 

\begin{table}[H]
\captionsetup{justification=raggedright,singlelinecheck=false}
\caption{\textbf{Tabla: caninos}}
\label{tab:dogs}
	\begin{tabular}{|m{3cm}|m{2cm}|m{5cm}|m{3cm}|}
	\hline
	\textbf{Campo} & \textbf{Tipo} & \textbf{Dato} & \textbf{Obligatorio} \\ 
	\hline
	\textbf{adopted} & booleano & Indica si el can ha sido adoptado & Sí \\ 
	\hline
	\textbf{age} & número & Edad &  Sí \\ 
	\hline
	\textbf{ageUds} & string & Unidades de la edad & Sí \\ 
	\hline
	\textbf{breed} & string & Raza &  Sí \\ 
	\hline
	\textbf{castrated} & booleano & Indica si está castrado & Sí \\ 
	\hline
	\textbf{color} & string & Color & Sí \\ 
	\hline
	\textbf{created} & marca de tiempo & Fecha y hora de cuándo se ha dado de alta &  Sí \\ 
	\hline
	\textbf{description} & string & Descripción & No \\ 
	\hline
	\textbf{favoriteBy} & array & IDs de usuarios que lo han marcado como favorito & No \\ 
	\hline
	\textbf{forAdoption} & booleano & Indica si está disponible para adopción & No \\ 
	\hline
	\textbf{forFoster} & booleano & Indica si está disponible para acogida & No \\ 
	\hline
	\textbf{gender} & string & Género & Sí \\ 
	\hline
	\textbf{name} & string & Nombre & Sí \\ 
	\hline
	\textbf{ownerId} & string & ID del propietario &  Sí \\ 
	\hline
	\textbf{profilePic} & string & URL de la imagen de perfil de la mascota & No \\ 
	\hline
	\textbf{weight} & decimal & Peso & Sí \\ 
	\hline
	\textbf{weightUds} & string & Unidades de peso &  Sí \\ 
	\hline
\end{tabular}
\end{table}

A continuación, se muestra un ejemplo real de un canino almacenado en la base de datos:

\begin{verbatim}
{
  "adopted": false,
  "age": 1,
  "ageUds": "Años",
  "breed": "Breed 2",
  "castrated": true,
  "color": "Color 1",
  "created": "23 de abril de 2024, 10:56:48 a.m. UTC+2",
  "description": "",
  "favoriteBy": [],
  "forAdoption": true,
  "forFoster": true,
  "gender": "female",
  "name": "Katrina",
  "ownerId": "S4wXwckahiSOLI0Q3exXDVRh2yl2",
  "profilePic": "",
  "weight": 1,
  "weightUds": "KG"
}
\end{verbatim}


\subsubsection{Guardado de direcciones}

Para todas las protectoras, es necesario almacenar la dirección en la que están ubicadas. Esta información es crucial para mostrar a los usuarios la ubicación de cada protectora y generar marcadores en el mapa correspondiente. Por un lado, se almacenarán todas las partes de la dirección, como la ciudad, la calle, el código postal, etc. Por otro lado, cuando un usuario introduzca su dirección, se generarán automáticamente las coordenadas de esa dirección para almacenarlas también, evitando tener que calcularlas cada vez que se carguen los mapas en la interfaz. Asimismo, cuando un usuario actualice su dirección, se actualizarán también sus coordenadas correspondientes en la base de datos.

Para todas las direcciónes se genera una entrada en el que su ID corresponde con el del usuario al que pertenece.

% Addresses
\begin{table}[H]
\captionsetup{justification=raggedright,singlelinecheck=false}
\caption{\textbf{Tabla: direcciones}}
\label{tab:Addresses}
	\begin{tabular}{|m{3cm}|m{2cm}|m{5cm}|m{3cm}|}
	\hline
	\textbf{Campo} & \textbf{Tipo} & \textbf{Dato} & \textbf{Obligatorio} \\ 
	\hline
	\textbf{city} & string & Ciudad & Sí \\ 
	\hline
	\textbf{country} & string & País (España siempre) & Sí \\ 
	\hline
	\textbf{province} &  string & Provincia & Sí \\ 
	\hline
	\textbf{street} &  string & Calle &  Sí \\ 
	\hline
	\textbf{street\_number} &  string & Número de la calle & No\\ 
	\hline
	\textbf{zipcode} & string & Código postal &  Sí \\ 
	\hline
	\textbf{lat} & decimal & Latitud & Sí \\ 
	\hline
	\textbf{lng} & decimal & Longitud &  Sí \\ 
	\hline
\end{tabular}
\end{table}

A continuación, se muestra un ejemplo real de una dirección almacenada en la base de datos:

\begin{verbatim}
{
  "city": "Murcia",
  "country": "España",
  "lat": 37.9879153,
  "lng": -1.1315578,
  "province": "Murcia",
  "street": "Calle Maestro Alonso",
  "street_number": "4",
  "zipcode": "30005"
}
\end{verbatim}

\subsubsection{Guardado de chats}

Para la tabla de chats, se ha decidido almacenar los IDs de los usuarios implicados. Además, se guarda un array de IDs que indica para quién está disponible el chat en ese momento. Al igual que en otras aplicaciones de mensajería, cuando se elimina un chat, éste simplemente desaparece para el usuario que lo borra, lo que modifica la propiedad de quién puede acceder al chat en ese momento. Sin embargo, los mensajes no se borran. Por ejemplo, si una persona A borra su chat con la persona B y luego vuelve a abrir un chat con esa misma persona, los mensajes anteriores seguirán estando disponibles. Almacenar aquí los IDs de los usuarios implicados permite que, si se proponen chats de grupo, sea más sencillo escalarlo, ya que la base de datos ya lo está soportando. Lo mismo ocurre con la propiedad que hace referencia a qué usuarios pueden ver el chat. Para todos los chats se genera un ID automático, al que se hace referencia desde la tabla de usuarios.

% Chats
\begin{table}[H]
\captionsetup{justification=raggedright,singlelinecheck=false}
\caption{\textbf{Tabla: chats}}
\label{tab:chats}
	\begin{tabular}{|m{3cm}|m{2cm}|m{5cm}|m{3cm}|}
	\hline
	\textbf{Campo} & \textbf{Tipo} & \textbf{Dato} & \textbf{Obligatorio} \\ 
	\hline
	\textbf{availableTo} & array & IDs de usuarios que tienen el chat visible en su página &  Sí \\ 
	\hline
	\textbf{lastMessage} & string & ID del último mensaje mandado en el chat &  Sí \\ 
	\hline
	\textbf{users} &  array & IDs de usuarios que intervienen en el chat &  Sí  \\ 
	\hline
\end{tabular}
\end{table}

A continuación se muestra un ejemplo real de un chat almacenado en la base de datos: 

\begin{verbatim}
{
  "availableTo": [
    "kuaKjc6XoiVot1rMeGqBySA5pE13",
    "buO4Vkazs3O2oVKetugxrvTfRw12"
  ],
  "lastMessage": "BaKmuuo8debUTdSNsvw0",
  "users": [
    "kuaKjc6XoiVot1rMeGqBySA5pE13",
    "buO4Vkazs3O2oVKetugxrvTfRw12"
  ]
}
\end{verbatim}

\subsubsection{Guardado de mensajes}

Aparte de almacenar toda la información de los chats, es necesario almacenar los mensajes que se envían a través de los mismos. Además de guardar el contenido del mensaje, se propone hacer una diferenciación por tipos. Concretamente, dos: \textbf{default} e \textbf{image}, que indican si un mensaje es texto o imagen, respectivamente. Esta definición de tipos ayudará a determinar cómo cargar el contenido en la interfaz. Para todos los mensajes, se genera un ID automático que se puede referenciar desde la tabla de chats.

% Mensajes
\begin{table}[H]
\captionsetup{justification=raggedright,singlelinecheck=false}
\caption{\textbf{Tabla: mensajes}}
\label{tab:messages}
	\begin{tabular}{|m{3.2cm}|m{2cm}|m{5cm}|m{3cm}|}
	\hline
	\textbf{Campo} & \textbf{Tipo} & \textbf{Dato} & \textbf{Obligatorio} \\ 
	\hline
	\textbf{chatId} & string & ID del chat &  Sí \\ 
	\hline
	\textbf{from} & string & ID del emisor del mensaje &  Sí \\ 
	\hline
	\textbf{messageContent} & cadena & Contenido del mensaje &  Sí \\ 
	\hline
	\textbf{read} & booleano & Indica si el recpetor a leído el mensaje &  Sí \\ 
	\hline
	\textbf{sent} & marca de tiempo & Fecha y hora en la que se mandó el mensaje & Sí \\ 
	\hline
	\textbf{to} & string & ID del receptor del mensaje & Sí \\ 
	\hline
	\textbf{type} & string & Tipo del mensaje & Sí \\ 
	\hline
	\end{tabular}
\end{table}

A continuación, se muestra un ejemplo real de un mensaje almacenado en la base de datos:

\begin{verbatim}
{
  "chatId": "Sm1Ji0chb9uhofa20VW5",
  "from": "kuaKjc6XoiVot1rMeGqBySA5pE13",
  "messageContent": "Hola, ¿estás?",
  "read": true,
  "sent": "6 de mayo de 2024, 1:57:30 a.m. UTC+2",
  "to": "buO4Vkazs3O2oVKetugxrvTfRw12",
  "type": "default"
}
\end{verbatim}


\subsubsection{Guardado de notificaciones}

En \textit{Firebase}, a las tablas se les llama colecciones y dentro de cualquier colección se pueden incluir subcolecciones. Este modelo de datos facilita evitar la necesidad de referenciar los datos a través de IDs de entradas de otras tablas. En el caso de la tabla de notificaciones, se ha definido que para cada usuario se genera una entrada en la tabla, y dentro de cada entrada se crea una colección para las peticiones. Al iniciar sesión, se añade un \textit{listener} que escucha todos los cambios de esta tabla, y cuando llega una nueva solicitud, se envía una notificación a través del canal correspondiente. Cuando el usuario receptor recibe la notificación, esta solicitud se marca como vista. El ID de cada entrada en la tabla de notificaciones corresponde al ID del usuario al que pertenece.

% notifications
\begin{table}[H]
\captionsetup{justification=raggedright,singlelinecheck=false}
\caption{\textbf{Tabla: notificaciones}}
\label{tab:notifications}
	\begin{tabular}{|m{3.2cm}|m{2cm}|m{5cm}|m{3cm}|}
	\hline
	\textbf{Campo} & \textbf{Tipo} & \textbf{Dato} & \textbf{Obligatorio} \\ 
	\hline
	\textbf{requests} & collection & Colección de peticiones &  Sí \\ 
	\hline
	\end{tabular}
\end{table}

A continuación, se muestra la estructura de la colección de peticiones:

% requests
\begin{table}[H]
\captionsetup{justification=raggedright,singlelinecheck=false}
\caption{\textbf{Tabla: peticiones}}
\label{tab:users}
	\begin{tabular}{|m{3.2cm}|m{2cm}|m{5cm}|m{3cm}|}
	\hline
	\textbf{Campo} & \textbf{Tipo} & \textbf{Dato} & \textbf{Obligatorio} \\ 
	\hline
	\textbf{body} & string & Contenido de la notificación &  Sí \\ 
	\hline
	\textbf{chatId} & string &  ID del chat desde el que se manda la notificación &  Sí \\ 
	\hline
	\textbf{new} & booleano & Indica si la notificación se ha mostrado o no &  Sí \\ 
	\hline
	\textbf{summary} & string & Vista previa de la notificación &  Sí \\ 
	\hline
	\textbf{title} & string & Titulo de la notificacion &  Sí \\ 
	\hline
	\textbf{userId} & string & Usuario que manda la notifiación &  Sí \\ 
	\hline
	\end{tabular}
\end{table}

A continuación, se muestra una petición real almacenada en la base de datos:

\begin{verbatim}
{
  "body": "Hola, ¿cómo estás?",
  "chatId": "Sm1Ji0chb9uhofa20VW5",
  "new": false,
  "summary": "Hola, ¿cómo estás?",
  "title": "Laura Vega Palacios",
  "userId": "kuaKjc6XoiVot1rMeGqBySA5pE13"
}
\end{verbatim}

\subsubsection{Almacenamiento de imágenes}

En cuanto al almacenamiento de imágenes, se ha utilizado el servicio que proporciona Firebase llamado \textit{Storage}. Este servicio permite almacenar imágenes, vídeos y audios y viene integrado de manera nativa con otros de sus servicios, como el de \textit{Authentication} mencionado anteriormente. Dentro del almacenamiento reservado para la aplicación, se ha organizado según la siguiente jerarquía de directorios:

\begin{itemize}[noitemsep]
	\item \textit{images}
	\begin{itemize}[noitemsep]
		\item \textit{chats}
		\item \textit{profilePics}
	\end{itemize}
\end{itemize}

Las imágenes de perfil de usuarios y perros se almacenan en el directorio \textit{profilePics}. Por otro lado, cuando se manda una imagen por chat, ésta se almacena en el directorio \textit{chats}. Para trabajar con las imágenes dentro de la aplicación, se pueden obtener los enlaces que se generan cuando se van almacenando dentro de \textit{Storage}. Estas URLs obtenidas se almacenan en las tablas definidas anteriormente. Si una de las imágenes es borrada, se tiene que borrar su referencia en la tabla de la base de datos y la imagen almacenada en \textit{Storage}.

\subsection{Prototipado de la interfaz de usuario}

El diseño de la Interfaz de Usuario (UI) se centra en crear una Experiencia de Usuario (UX) intuitiva y atractiva. Inicialmente, se define un estilo y tema que se debe mantener a lo largo de todos los bocetos o prototipos a generar. Es importante tener en cuenta que el diseño sea \textit{responsive} para que se adapte adecuadamente a todos los dispositivos móviles.

\subsubsection{Tema}

Para los colores principales de la aplicación, se opta por escoger el color morado. El color morado dentro de UX puede evocar diferentes sensaciones en el usuario. Muchas marcas de lujo utilizan este color para sus logos, aunque también puede evocar sentimientos de calma o relajación según las tonalidades que se utilicen. Teniendo el morado como color principal, se añade a la paleta su color análogo, el azul. Para finalizar los colores principales, a la paleta se añade el color complementario del azul, que sería el naranja. Estos tres colores conformarán los elementos de la aplicación; el morado se utilizará para la mayoría de los elementos, indicando que son importantes. El azul se mezclará con el morado en algunos elementos para hacer más dinámicos los fondos y utilizarlo en elementos no tan relevantes. Por último, el naranja será un color que se utilice como acento o para llamar la atención del usuario, junto con el naranja se podrán utilizar algunos de sus colores análogos como el amarillo o el rojo. Para los fondos de la aplicación se añade también un blanco roto a la paleta de colores propuesta.

\begin{figure}[H]
	\begin{center}
		{\includegraphics[width=7cm]{design/Colors.jpg}\par}
		\caption{Paleta de colores}
	\end{center}
\end{figure}

Como fuente, se opta por escoger la fuente \textit{Nunito Sans}, que proporciona Google. Se elige porque es una fuente bastante estándar y legible para cualquier tipo de usuario.

\begin{figure}[H]
	\begin{center}
		{\includegraphics[width=7cm]{design/NunitoFont.png}\par}
		\caption{Fuente Nunito Sans}
	\end{center}
\end{figure}

\subsubsection{Nombre}

Como nombre de la aplicación,  se tiene en cuenta que inicialmente la aplicación está dirigida a la adopción de perros, así que se propone un juego de palabras. Uniendo las palabras \textit{Adoption} y \textit{Dog}, que son la traducción de las palabras adopción y perro, conformando el nombre: \textbf{\textit{Adogption}}.

\subsubsection{Logo e iconos}

La última parte relacionada con el branding en este proyecto es la de generación de logos e iconos. En esta sección se incluye tanto el logo de la aplicación como los iconos que se han hecho para incluir dentro de la aplicación. Algunos de los iconos podrían finalmente no utilizarse.

La idea del logo es que se perciba claramente que la aplicación va sobre perros. Los corgis son una de las razas más populares en el mundo; destacan por ser una raza muy amigable y muy querida dentro de la cultura pop, lo que los convierte en la cara perfecta para llamar la atención. Además del logo principal, se ha diseñado un logo animado para incluir en pantallas de carga u otros lugares. Por último, se han hecho algunos iconos por defecto, que serán utilizados en perfiles o listas.


\begin{figure}[H]
   	\begin{subfigure}{0.48\textwidth}
		\begin{center}
			{\includegraphics[width=6cm]{logo/Logo1.png}\par}
			\caption{Logo.}
		\end{center}  
	\end{subfigure}\hfill
   	\begin{subfigure}{0.48\textwidth}
		\begin{center}
			{\includegraphics[width=7cm]{logo/ADOGPTIONFIXED.png}\par}
			\caption{Logo animado.}
		\end{center}  
	\end{subfigure}\hfill
	\caption{Logos}
\end{figure}


\begin{figure}[H]
   	\begin{minipage}{0.48\textwidth}
		\begin{center}
			{\includegraphics[width=6cm]{logo/EMPTYUSER.png}\par}
			\caption{Icono por defecto para usuario.}
		\end{center}  
	\end{minipage}\hfill
   	\begin{minipage}{0.48\textwidth}
		\begin{center}
			{\includegraphics[width=6cm]{logo/EMPTYDOG.png}\par}
			\caption{Icono por defecto para perro.}
		\end{center}  
	\end{minipage}\hfill
\end{figure}


\begin{figure}[H]
   	\begin{minipage}{0.48\textwidth}
		\begin{center}
			{\includegraphics[width=6cm]{logo/Adoptar.png}\par}
			\caption{Icono para listas de adopción.}
		\end{center}  
	\end{minipage}\hfill
   	\begin{minipage}{0.48\textwidth}
		\begin{center}
			{\includegraphics[width=6cm]{logo/Acoger.png}\par}
			\caption{Icono para listas de acogida.}
		\end{center}  
	\end{minipage}\hfill
\end{figure}

\begin{figure}[H]
   	\begin{minipage}{0.48\textwidth}
		\begin{center}
			{\includegraphics[width=6cm]{logo/USERNV.png}\par}
			\caption{Icono usuario no verificado.}
		\end{center}  
	\end{minipage}\hfill
   	\begin{minipage}{0.48\textwidth}
		\begin{center}
			{\includegraphics[width=6cm]{logo/USERV.png}\par}
			\caption{Icono usuario verificado.}
		\end{center}  
	\end{minipage}\hfill
\end{figure}

\begin{figure}[H]
   	\begin{minipage}{0.48\textwidth}
		\begin{center}
			{\includegraphics[width=6cm]{logo/Protectora.png}\par}
			\caption{Icono hueso.}
		\end{center}  
	\end{minipage}\hfill
   	\begin{minipage}{0.48\textwidth}
		\begin{center}
			{\includegraphics[width=6cm]{logo/Casa.png}\par}
			\caption{Icono casa.}
		\end{center}  
	\end{minipage}\hfill
\end{figure}


\begin{figure}[H]
   	\begin{subfigure}{0.48\textwidth}
		\begin{center}
			{\includegraphics[width=6cm]{logo/ALLUSER.png}\par}
			\caption{Icono en morado.}
		\end{center}  
	\end{subfigure}\hfill
   	\begin{subfigure}{0.48\textwidth}
		\begin{center}
			{\includegraphics[width=6cm]{logo/USER.png}\par}
			\caption{Icono en azul.}
		\end{center}  
	\end{subfigure}\hfill
	\caption{Iconos usuarios}
\end{figure}

\begin{figure}[H]
   	\begin{subfigure}{0.48\textwidth}
		\begin{center}
			{\includegraphics[width=6cm]{logo/Pata1.png}\par}
			\caption{Icono en naranja.}
		\end{center}  
	\end{subfigure}\hfill
   	\begin{subfigure}{0.48\textwidth}
		\begin{center}
			{\includegraphics[width=6cm]{logo/Pata2.png}\par}
			\caption{Icono en morado.}
		\end{center}  
	\end{subfigure}\hfill
   	\begin{subfigure}{0.48\textwidth}
		\begin{center}
			{\includegraphics[width=6cm]{logo/Pata3.png}\par}
			\caption{Icono en azul.}
		\end{center}  
	\end{subfigure}\hfill
	\caption{Iconos patas}
\end{figure}


\newpage
\subsubsection{Diseños}

En esta sección se incluyen todos los diseños realizados en \textit{Miró} para las pantallas de la aplicación. Entre los diseños se encuentran todas las páginas principales de la aplicación y componentes relevantes. En la sección de implementación se especificará la función de cada una de las pantallas.


\begin{figure}[H]
   	\begin{minipage}{0.48\textwidth}
		\begin{center}
			{\includegraphics[height=8cm]{design/SplashScreen.jpg}\par}
			\caption{Pantalla de carga.}
			\medskip
		\end{center}  
	\end{minipage}\hfill
   	\begin{minipage}{0.48\textwidth}
		\begin{center}
			{\includegraphics[height=8cm]{design/Login.jpg}\par}
			\caption{Inicio de sesión.}
			\medskip
		\end{center}  
	\end{minipage}\hfill
\end{figure}


\begin{figure}[H]
   	\begin{minipage}{0.48\textwidth}
		\begin{center}
			{\includegraphics[height=8cm, width=4cm]{design/UserRegister.jpg}\par}
			\caption{Registro/edición de usuario.}
			\medskip
		\end{center}  
	\end{minipage}\hfill
   	\begin{minipage}{0.48\textwidth}
		\begin{center}
			{\includegraphics[height=8cm, width=4cm]{design/CompanyRegister.jpg}\par}
			\caption{Registro/edición de protectora.}
			\medskip
		\end{center}  
	\end{minipage}\hfill
\end{figure}

\begin{figure}[H]
   	\begin{minipage}{0.48\textwidth}
		\begin{center}
			{\includegraphics[height=8cm, width=4cm]{design/UserPage.jpg}\par}
			\caption{Inicio usuario.}
			\medskip
			\small

		\end{center}  
	\end{minipage}\hfill
   	\begin{minipage}{0.48\textwidth}
		\begin{center}
			{\includegraphics[height=8cm, width=4cm]{design/CompanyPage.jpg}\par}
			\caption{Inicio protectora.}
			\medskip			
		\end{center}  
	\end{minipage}\hfill
\end{figure}

\begin{figure}[H]
   	\begin{minipage}{0.48\textwidth}
		\begin{center}
			{\includegraphics[height=8cm, width=4cm]{design/DogList.jpg}\par}
			\caption{Lista de perros.}
			\medskip
		\end{center}  
	\end{minipage}\hfill
   	\begin{minipage}{0.48\textwidth}
		\begin{center}
			{\includegraphics[height=8cm, width=4cm]{design/UserList.jpg}\par}
			\caption{Lista de usuarios.}
			\medskip
		\end{center}  
	\end{minipage}\hfill
\end{figure}


\begin{figure}[H]
   	\begin{minipage}{0.48\textwidth}
		\begin{center}
			{\includegraphics[height=8cm, width=4cm]{design/UserProfile.jpg}\par}
			\caption{Perfil de usuario/administrador.}
			\medskip

		\end{center}  
	\end{minipage}\hfill
   	\begin{minipage}{0.48\textwidth}
		\begin{center}
			{\includegraphics[height=8cm, width=4cm]{design/CompanyProfile.jpg}\par}
			\caption{Perfil de protectora. }
			\medskip
		\end{center}  
	\end{minipage}\hfill
\end{figure}

\begin{figure}[H]
   	\begin{minipage}{0.48\textwidth}
		\begin{center}
			{\includegraphics[height=8cm, width=4cm]{design/DogProfile.jpg}\par}
			\caption{Perfil de perro.}
			\medskip
		\end{center}  
	\end{minipage}\hfill
   	\begin{minipage}{0.48\textwidth}
		\begin{center}
			{\includegraphics[height=8cm, width=4cm]{design/ShareAction.jpg}\par}
			\caption{Compartir perro.}
			\medskip
		\end{center}  
	\end{minipage}\hfill
\end{figure}

\begin{figure}[H]
   	\begin{minipage}{0.48\textwidth}
		\begin{center}
			{\includegraphics[height=8cm, width=4cm]{design/MapPage.jpg}\par}
			\caption{Página de mapa y lista.}
			\medskip
		\end{center}  
	\end{minipage}\hfill
   	\begin{minipage}{0.48\textwidth}
		\begin{center}
			{\includegraphics[height=8cm, width=4cm]{design/MapAction.jpg}\par}
			\caption{Acción de redirigir a perfil.}
			\medskip
		\end{center}  
	\end{minipage}\hfill
\end{figure}

\begin{figure}[H]
   	\begin{minipage}{0.48\textwidth}
		\begin{center}
			{\includegraphics[height=8cm, width=4cm]{design/RegisterDog.jpg}\par}
			\caption{Registro/edición canino.}
			\medskip
		\end{center}  
	\end{minipage}\hfill
   	\begin{minipage}{0.48\textwidth}
		\begin{center}
			{\includegraphics[height=8cm, width=4cm]{design/SideMenu.jpg}\par}
			\caption{Menú lateral.}
			\medskip
		\end{center}  
	\end{minipage}\hfill
\end{figure}

\begin{figure}[H]
   	\begin{minipage}{0.48\textwidth}
		\begin{center}
			{\includegraphics[height=8cm, width=4cm]{design/AdminPage.jpg}\par}
			\caption{Inicio administrador.}
			\medskip
		\end{center}  
	\end{minipage}\hfill
   	\begin{minipage}{0.48\textwidth}
		\begin{center}
			{\includegraphics[height=8cm, width=4cm]{design/VerifyAction.jpg}\par}
			\caption{Pop-up verificación.}
			\medskip
		\end{center}  
	\end{minipage}\hfill
\end{figure}

\begin{figure}[H]
   	\begin{minipage}{0.48\textwidth}
		\begin{center}
			{\includegraphics[height=8cm, width=4cm]{design/Notification.jpg}\par}
			\caption{Notificaciones.}
			\medskip
		\end{center}  
	\end{minipage}\hfill
   	\begin{minipage}{0.48\textwidth}
		\begin{center}
			{\includegraphics[height=8cm, width=4cm]{design/Messages.jpg}\par}
			\caption{Página de mensajes.}
			\medskip
		\end{center}  
	\end{minipage}\hfill
\end{figure}

\begin{figure}[H]
   	\begin{minipage}{0.48\textwidth}
		\begin{center}
			{\includegraphics[height=8cm, width=4cm]{design/PasswordRecovery.jpg}\par}
			\caption{Recuperar contraseña.}
			\medskip
		\end{center}  
	\end{minipage}\hfill
\end{figure}


% Implementación
\newpage
\section{Implementación}

En esta sección se incluyen las clases y pantallas finales, que son resultado del desarrollo y de la posterior corrección de errores.

\subsection{Módulos y clases}

Durante el desarrollo se han implementado diferentes módulos, cada uno con una finalidad específica. Mantener los componentes granulados en módulos y submódulos asegura una mejor escalabilidad y mantenibilidad de la aplicación y facilita tareas como refactorizaciones y cambios de estilos, además de evitar muchas repeticiones de código.


%Diagrama de modulos y sus dependencias
\begin{figure}[H]
	\begin{center}
		{\includegraphics[width=13cm]{diagram/MODULES.png}\par}
		\caption{Módulos y sus dependencias.}
	\end{center}
\end{figure}


\newpage
\subsection*{appbar}


Este módulo se utiliza para generar las diferentes \textit{AppBar} que se utilizan en las pantallas de la aplicación. Una \textit{AppBar} es el componente que se coloca en la parte superior de la pantalla y puede contener botones u otros componentes tales como barras de búsqueda o imágenes. La única clase que contiene el módulo se encarga de instanciar el componente en las diferentes pantallas según el tipo de barra que se necesite.

\begin{figure}[H]
	\begin{center}
		{\includegraphics[]{diagram/AppBar.png}\par}
		\caption{Módulo \textit{appbar}.}
	\end{center}
\end{figure}


\subsection*{chats}

Aquí se incluyen todas las clases que necesita un chat para funcionar correctamente.
\begin{itemize}[noitemsep]
	\item \textbf{ChatInput}: Componente que permite a los usuarios escribir mensajes y adjuntar imágenes a los chats.
	\item \textbf{ChatRoomPage}: Interfaz del chat.
	\item \textbf{ChatsPage}:  Es la página principal que contiene todos los chats, contiene todos los chats que un usuario tiene en ese momento, son una lista de \textit{ChatWidget}.
	\item \textbf{ChatWidget}: Este componente muestra una previsualización del chat, que incluye la información del usuario, el último mensaje y la fecha y hora en la que se envió. Además, incluye el número de mensajes sin leer del chat.
	\item \textbf{ChatModel}: Clase que se encarga de generar y manejar modelos de chat que se generan a partir de toda la información de la base de datos relacionada con los chats.
\end{itemize}

\begin{figure}[H]
	\begin{center}
		{\includegraphics[width=0.7\linewidth]{diagram/Chats.png}\par}
		\caption{Módulo \textit{chats}.}
	\end{center}
\end{figure}


\subsection*{common}

En este módulo se definen todos los componentes básicos que incluyen las diferentes pantallas generadas. Además, se definen diferentes submódulos para cada uno de los componentes creados. Todos los submódulos contienen una clase base en la que se definen los estilos y funcionalidades básicas, y el resto de las clases extienden estas clases base. Es importante destacar que, aunque se hayan creado clases propias para los componentes, todas ellas instancian componentes de las bibliotecas por defecto de Flutter, como la biblioteca \textit{material}.

\begin{itemize}[noitemsep]
	\item \textbf{button}: En este submódulo se definen todos los botones utilizados en diversos componentes y pantallas de la aplicación. Estos botones tienen diferentes funcionalidades, como la redirección a otras pantallas, la actualización de datos o la visualización de pop-ups.
	\item \textbf{card}: Una tarjeta es un componente similar a un contenedor pero con un aspecto más atractivo. Se ha creado un componente base que se reutiliza en diferentes widgets y que contiene este módulo.
	\item \textbf{checkbox}: Contiene los diferentes checkboxes desarrollados, junto con su comportamiento asociado.
	\item \textbf{container}: Define un contenedor básico con un estilo concreto para reutilizarlo en diferentes lugares.
	\item \textbf{dialog}: Agrupa todos los pop-ups desarrollados y su comportamiento correspondiente.
	\item \textbf{drawer}: Contiene todos los menús laterales y los componentes asociados. Estos menús incluyen imágenes de perfil y botones que redirigen a otras pantallas.
	\item \textbf{dropdown}: Incluye los diferentes dropdowns utilizados en la aplicación, algunos de los cuales tienen la funcionalidad de búsqueda de opciones.
	\item \textbf{form}: Contiene todos los formularios utilizados en la aplicación.
	\item \textbf{icon}: Agrupa iconos básicos con funcionalidad. Por ahora, solo incluye el icono de favoritos.
	\item \textbf{images}: Incluye clases que proporcionan funcionalidades para seleccionar imágenes del dispositivo y componentes para facilitar la inserción de imágenes en la aplicación.
	\item \textbf{input}: Contiene componentes que permiten a los usuarios introducir información de diversos tipos, como texto, números o componentes compuestos que contienen dropdowns o checkboxes.
	\item \textbf{maps}: Incluye toda la funcionalidad relacionada con los mapas, tanto la página de mapas como los mapas que se pueden insertar en diferentes partes de la aplicación.
	\item \textbf{padding}: Define un padding que se reutiliza en toda la aplicación.
	\item \textbf{picker}: Incluye componentes relacionados con la selección de direcciones en mapas u opciones similares.
	\item \textbf{searchbar}: Define las barras de búsqueda implementadas en la aplicación, junto con su funcionalidad.
	\item \textbf{swiper}: Por ahora, solo contiene el swiper utilizado para mostrar las adopciones recientes.
	\item \textbf{text}: Contiene clases que definen textos utilizados en la aplicación, como títulos, subtítulos y avisos.
	\item \textbf{widget}: Incluye componentes comunes compuestos por otros componentes comunes.
\end{itemize}


\begin{figure}[H]
	\begin{center}
		{\includegraphics[width=0.8\linewidth]{diagram/Common.png}\par}
		\caption{Módulo  \textit{common}.}
	\end{center}
\end{figure}

\subsection*{condition}

Este módulo se encarga de gestionar los filtros de las listas de una manera más dinámica.

\begin{itemize}[noitemsep]
	\item \textbf{Basic}: Define el componente básico para añadir filtros a las listas, encargándose de abrir el pop-up con las opciones del filtro.
	\item \textbf{Gender}: Define el filtro para el género.
	\item \textbf{Weight}: Define el filtro para el peso.
	\item \textbf{ConditionModel}: Clase que se encarga de generar y manejar modelos de condiciones a partir de la información proporcionada por la aplicación o el usuario. Esta clase además proporciona un método que se encarga de añadir todos los filtros que haya aplicados en ese momento a la consulta.
\end{itemize}

\begin{figure}[H]
	\begin{center}
		{\includegraphics[]{diagram/Condition.png}\par}
		\caption{Módulo \textit{condition}.}
	\end{center}
\end{figure}

\subsection*{contact}

Aunque este módulo solo contiene una clase que se encarga de mostrar toda la información de contacto con los administradores, se definió por separado para poder añadir más clases en futuras iteraciones.

\begin{figure}[H]
	\begin{center}
		{\includegraphics[]{diagram/Contact.png}\par}
		\caption{Módulo \textit{contact}.}
	\end{center}
\end{figure}


\subsection*{db}

Este módulo es el encargado de comunicarse directamente con la base de datos de Firebase. Contiene diferentes clases para las distintas entidades que existen y para el almacenamiento.

\begin{itemize}[noitemsep]
	\item \textbf{DBBase}: Clase abstracta que contiene los métodos básicos que deben incluir el resto de clases.
	\item \textbf{DBAddresses}: Clase que maneja las direcciones y coordenadas.
	\item \textbf{DBChats}: Clase que maneja toda la información relacionada con chats.
	\item \textbf{DBDogs}: Clase que maneja toda la información relacionada con perros y su información.
	\item \textbf{DBFavorites}: Clase que se encarga exclusivamente de actualizar la información de los favoritos dentro de la tabla de perros.
	\item \textbf{DBMessages}: Clase que maneja toda la información relacionada con mensajes.
	\item \textbf{DBNotifications}: Clase que maneja toda la información relacionada con notificaciones.
	\item \textbf{DBStorage}: Clase que se encarga de insertar, actualizar y eliminar todas las imágenes relacionadas con alguna de las entidades.
	\item \textbf{DBUsers}: Clase que maneja toda la información relacionada con los usuarios.
\end{itemize}

\begin{figure}[H]
	\begin{center}
		{\includegraphics[width=0.7\linewidth]{diagram/DB.png}\par}
		\caption{Módulo  \textit{db}.}
	\end{center}
\end{figure}


\subsection*{dog}

Módulo que contiene las clases principales relacionadas con los perros.

\begin{itemize}[noitemsep]
	\item \textbf{AvailableDogsPage}: Pantalla de lista de perros que contiene los filtros y una barra de búsqueda. Se encarga de manejar las consultas y mostrar los resultados.
	\item \textbf{DogFeature}: Widget que se instancia con la información de alguno de los atributos del perro para mostrarlo en el perfil.
	\item \textbf{DogProfile}: Pantalla que contiene toda la información de un perro, incluyendo el botón de favoritos y compartir.
	\item \textbf{DogModel}: Clase que se encarga de generar y manejar modelos de perros a partir de la información en la base de datos.
\end{itemize}

\begin{figure}[H]
	\begin{center}
		{\includegraphics[width=0.8\linewidth]{diagram/Dog.png}\par}
		\caption{Módulo  \textit{dog}.}
	\end{center}
\end{figure}


\subsection*{home}

Se incluyen todas las pantallas de inicio definidas para los diferentes roles.

\begin{itemize}[noitemsep]
	\item \textbf{AdminHome}: Pantalla de inicio para administradores.
	\item \textbf{MyHome}: Pantalla de inicio para usuarios.
	\item \textbf{MyHomePageCompany}: Pantalla de inicio para protectoras.
\end{itemize}

\begin{figure}[H]
	\begin{center}
		{\includegraphics[width=0.8\linewidth]{diagram/Home.png}\par}
		\caption{Módulo  \textit{home}.}
	\end{center}
\end{figure}


\subsection*{init}

La clase contenida en este módulo se encarga de manejar la pantalla que se debe mostrar cuando se abre la aplicación, además de inicializar todos los servicios que lo requieran. La pantalla mostrará la pantallla de carga y, dependiendo de si hay un usuario iniciado, se redigirá a la pantalla de inicio del rol correspondiente o a la pantalla de inicio de sesión.

\begin{figure}[H]
	\begin{center}
		{\includegraphics[]{diagram/Init.png}\par}
		\caption{Módulo  \textit{init}.}
	\end{center}
\end{figure}


\subsection*{legal}

Este módulo solo contiene una pantalla que contiene toda la información legal relacionada con las adopciones y los usos de la aplicación.

\begin{figure}[H]
	\begin{center}
		{\includegraphics[]{diagram/Legal.png}\par}
		\caption{Módulo  \textit{legal}.}
	\end{center}
\end{figure}


\subsection*{login}


Contiene la página para iniciar sesión y la de recuperación de contraseña.

\begin{figure}[H]
	\begin{center}
		{\includegraphics[width=0.8\linewidth]{diagram/Login.png}\par}
		\caption{Módulo  \textit{login}.}
	\end{center}
\end{figure}


\subsection*{message}

En este módulo se definen los widgets relacionados con los mensajes que se utilizan dentro de las salas de chat.

\begin{itemize}[noitemsep]
	\item \textbf{DateMessageDivider}: Este widget se coloca entre mensajes de días diferentes para indicar la fecha de los mensajes siguientes.
	\item \textbf{MessageWidget}: Este widget muestra el contenido del mensaje, que puede ser texto o imagen, y también incluye un ícono que indica si el mensaje ha sido leído o no.
	\item \textbf{MessageModel}: Clase que se encarga de generar y manejar modelos de mensajes a partir de la información en la base de datos.
\end{itemize}


\begin{figure}[H]
	\begin{center}
		{\includegraphics[width=0.8\linewidth]{diagram/Message.png}\par}
		\caption{Módulo \textit{message}.}
	\end{center}
\end{figure}


\subsection*{orderby}

Este módulo contiene el modelo para añadir la cláusula de orden a las consultas.

\begin{figure}[H]
	\begin{center}
		{\includegraphics[]{diagram/OrderBy.png}\par}
		\caption{Módulo  \textit{orderby}.}
	\end{center}
\end{figure}


\subsection*{register}

Contiene todas las páginas relacionadas con los registros dentro de la app. Cada una de las pantallas incluye un formulario específico que se encarga de todas las validaciones.

\begin{itemize}[noitemsep]
	\item \textbf{RegisterAsCompanyPage}: Pantalla de registro de protectora.
	\item \textbf{RegisterAsUserPage}: Pantalla de registro de usuario.
	\item \textbf{RegisterDog}: Pantralla de registro de perro.
	\item \textbf{RegisterPage}:  Pantalla que ofrece las opciones para registrarse como protectora o usuario.
\end{itemize}

\begin{figure}[H]
	\begin{center}
		{\includegraphics[width=0.8\linewidth]{diagram/Register.png}\par}
		\caption{Módulo  \textit{register}.}
	\end{center}
\end{figure}

\subsection*{section}

Contiene diferentes widgets que definen secciones que se reutilizan en algunas de las pantallas.

\begin{itemize}[noitemsep]
	\item \textbf{AdminCategorySection}: Incluye la sección principal de la página de inicio de administrador.
	\item \textbf{CompanyCategorySection}: Incluye la sección principal de la página de inicio de protectora.
	\item \textbf{RecentAdoptionsSection}:  Define la sección de adopciones recientes que se usa en diferentes páginas de inicio.
	\item \textbf{UserCategorySection}:  Incluye la sección principal de la página de inicio de usuario.
	\item \textbf{UserFavoritesSection}:  Define la sección de favoritos de un usuario específico.
	\item \textbf{UserOwnedDogsSection}:  Define la sección de perros que son de un usuario específico.
\end{itemize}


\begin{figure}[H]
	\begin{center}
		{\includegraphics[width=0.8\linewidth]{diagram/Section.png}\par}
		\caption{Módulo  \textit{section}.}
	\end{center}
\end{figure}


\subsection*{services}

Se incluyen clases que manejan los servicios que proporciona Flutter o Firebase demás de otras clases que son servicios creados específicamente para la aplicación.

\begin{itemize}[noitemsep]
	\item \textbf{AddressService}: Servicio para manejar los modelos de las direcciones y las coordenadas.
	\item \textbf{AwesomeNotificationService}: Clase que maneja el servicio de notificaciones. 
	\item \textbf{GeocodingService}: Clase que maneja el servicio de geocoding que proporciona Google dentro de la app.
	\item \textbf{RoutingService}:  Servicio para manejar las rutas de la aplicación.
	\item \textbf{Auth - AuthUsers}:  Clase que maneja el servicio de autenticación que proporciona Firebase.
\end{itemize}



\begin{figure}[H]
	\begin{center}
		{\includegraphics[width=0.8\linewidth]{diagram/Services.png}\par}
		\caption{Módulo  \textit{services}.}
	\end{center}
\end{figure}

\subsection*{update}

Contiene las páginas que incluyen formularios para la actualización de datos de usuarios y de perros.

\begin{figure}[H]
	\begin{center}
		{\includegraphics[]{diagram/Update.png}\par}
		\caption{Módulo  \textit{update}.}
	\end{center}
\end{figure}



\subsection*{user}

Módulo que contiene todas las clases relacionadas con los usuarios y protectoras.

\begin{itemize}[noitemsep]
	\item \textbf{CurrentUser}: Clase singleton que se encarga de manejar todos los datos relacionados con el usuario iniciado a lo largo de la aplicación.
	\item \textbf{LoggedUserModel}: Clase para manejar el modelo del usuario iniciado.
	\item \textbf{UserModel}: Clase que se encarga de generar y manejar modelos de usuarios a partir de la información en la base de datos. 
	\item \textbf{ProfilePage}:  Pantalla que contiene toda la información personal del usuario. Puede contener un listado de perros y un botón para abrir una página de mapa dependiendo del rol de usuario.
	\item \textbf{UserMapWidget}: Elemento que se instancia en las listas usadas en los mapas con toda la información del usuario.
	\item \textbf{UsersPage}: Pantalla de la lista de usuarios que contiene además la barra de búsqueda. Se encarga de manejar las consultas y mostrar los resultados.
\end{itemize}

\begin{figure}[H]
	\begin{center}
		{\includegraphics[width=0.8\linewidth]{diagram/User.png}\par}
		\caption{Módulo  \textit{user}.}
	\end{center}
\end{figure}

\newpage
\subsection{Pantallas}

En esta sección se exponen todas las pantallas resultantes del desarrollo.

%Splashscreen
\subsection*{Pantalla de carga}

La pantalla de carga es una pantalla de transición que se utiliza justo después de iniciar sesión, mientras se carga toda la información del usuario.

\begin{figure}[H]
	\begin{center}
		{\includegraphics[width=6cm]{app/Splashscreen.png}\par}
		\caption{Pantalla de carga}
	\end{center}
\end{figure}


%Login
\newpage
\subsection*{Inicio de sesión}

La pantalla de inicio de sesión muestra un formulario básico para introducir el correo electrónico y la contraseña. Este formulario indicará si hay errores durante el intento de iniciar sesión. El botón de inicio de sesión maneja toda la autenticación mediante el servicio de autenticación de Firebase.

\begin{figure}[H]
   	\begin{subfigure}{0.48\textwidth}
		\begin{center}
			{\includegraphics[width=6cm]{app/Login.png}\par}
			\caption{Inicio de sesión.}
		\end{center}  
	\end{subfigure}\hfill
   	\begin{subfigure}{0.48\textwidth}
		\begin{center}
			{\includegraphics[width=6cm]{app/LoginWithErrors.png}\par}
			\caption{Inicio de sesión con errores}
		\end{center}  
	\end{subfigure}\hfill
	\caption{Pantalla de inicio de sesión.}\label{fig:login}
\end{figure}



%Recover password
\newpage
\subsection*{Recuperación de contraseña}

Esta pantalla contiene un formulario que solo incluye un campo para introducir el correo electrónico. Si el correo electrónico está registrado, el servicio de autenticación de Firebase se encargará de enviarle un correo electrónico con las instrucciones para cambiar la contraseña.

\begin{figure}[H]
	\begin{center}
		{\includegraphics[width=6cm]{app/RecoverPassword.png}\par}
		\caption{Pantalla de recuperación de contraseña.}
	\end{center}
\end{figure}

%Register
\newpage
\subsection*{Registro}

Si se pulsa el botón \textit{Regístrate}, que se incluye en la pantalla de inicio de sesión mostrada en la Figura \ref{fig:login}, se redirige a una pantalla que muestran dos botones que llevan a los formularios para registrarse como usuario normal o como protectora respectivamente. Ambos formularios de registro piden los mismos datos a excepción de la dirección, que solo se requiere en el caso de las protectoras.

\begin{figure}[H]
	\begin{center}
		{\includegraphics[width=6cm]{app/Register.png}\par}
		\caption{Página de opciones de registro.}
	\end{center}  
\end{figure}

\begin{figure}[H]
   	\begin{subfigure}{0.48\textwidth}
		\begin{center}
			{\includegraphics[width=6cm]{app/RegisterUserEmpty.png}\par}
			\caption{Registro como usuario.}
		\end{center}  
	\end{subfigure}\hfill
   	\begin{subfigure}{0.48\textwidth}
		\begin{center}
			{\includegraphics[width=6cm]{app/RegisterCompanyAddress.png}\par}
			\caption{Registro como protectora.}
		\end{center}  
	\end{subfigure}\hfill
	\caption{Pantallas de registro.}
\end{figure}


\begin{figure}[H]
   	\begin{subfigure}{0.48\textwidth}
		\begin{center}
			{\includegraphics[width=6cm]{app/RegisterUserError.png}\par}
			\caption{Formulario base con errores.}
		\end{center}  
	\end{subfigure}\hfill
   	\begin{subfigure}{0.48\textwidth}
		\begin{center}
			{\includegraphics[width=6cm]{app/AddressError.png}\par}
			\caption{Formulario con dirección con errores.}
		\end{center}  
	\end{subfigure}\hfill
	\caption{Pantallas de registro con errores.}
\end{figure}


El componente más complejo de estos formularios es la barra de búsqueda de direcciones. Este componente se ha desarrollado en base a un componente proporcionado por el paquete \href{https://pub.dev/packages/search\_map\_location}{\textit{search\_map\_location}} \cite{search_map_location}, que proporciona la funcionalidad básica para buscar direcciones con la API de Google Maps. A este componente se le ha añadido el comportamiento de autocompletar los campos del formulario una vez se selecciona una dirección, además de ajustar todos los estilos según los de la aplicación.

\begin{figure}[H]
   	\begin{subfigure}{0.48\textwidth}
		\begin{center}
			{\includegraphics[width=6cm]{app/AddressSearcherOnSearch.png}\par}
			\caption{Búsqueda de direcciones.}
		\end{center}  
	\end{subfigure}\hfill
   	\begin{subfigure}{0.48\textwidth}
		\begin{center}
			{\includegraphics[width=6cm]{app/AddressSearcherAutoComplete.png}\par}
			\caption{Autocompletar campos de direcciones.}
		\end{center}  
	\end{subfigure}\hfill
	\caption{Barra de búsqueda de direcciones.}
\end{figure}


% Home
\newpage
\subsection*{Pantallas de inicio}

Para cada uno de los roles definidos, se ha creado una pantalla de inicio específica. Ambas pantallas incluyen bastante contenido en común, como la sección de adopciones recientes o la sección de favoritos. Se incluyen listas para mostrar los perros marcados para adoptar o acoger en ambas pantallas, con la diferencia de que una protectora solo verá sus propios perros en esas listas. 

\begin{figure}[H]
   	\begin{subfigure}{0.48\textwidth}
		\begin{center}
			{\includegraphics[width=6cm]{app/UserHome.png}\par}
			\caption{Inicio de usuario.}
		\end{center}  
	\end{subfigure}\hfill
   	\begin{subfigure}{0.48\textwidth}
		\begin{center}
			{\includegraphics[width=6cm]{app/CompanyHome.png}\par}
			\caption{Inicio de protectora.}
		\end{center}  
	\end{subfigure}\hfill
	\caption{Pantallas de inicio por rol.}
\end{figure}


Para cada uno de los roles también se han implementado menús laterales con distintas opciones.

\begin{figure}[H]
   	\begin{subfigure}{0.48\textwidth}
		\begin{center}
			{\includegraphics[width=6cm]{app/UserSideMenu.png}\par}
			\caption{Menú lateral de usuario.}
		\end{center}  
	\end{subfigure}\hfill
   	\begin{subfigure}{0.48\textwidth}
		\begin{center}
			{\includegraphics[width=6cm]{app/CompanySideMenu.png}\par}
			\caption{Menú lateral de protectora.}
		\end{center}  
	\end{subfigure}\hfill
	\caption{Menús laterales por rol.}
\end{figure}


Si una protectora inicia sesión sin estar verificada por un administrador, se muesta una pantalla informándole que tiene que esperar la verificación para poder seguir usando la aplicación. Aunque la protectora no pueda subir perros o mensajear, si podrá abrir un chat con un administrador desde la página de contacto.


\begin{figure}[H]
	\begin{center}
		{\includegraphics[width=6cm]{app/UnverifiedPage.png}\par}
		\caption{Pantalla de protectora no verificada.}
	\end{center}  
\end{figure}



% Listas
\newpage
\subsection*{Listas}

A lo largo de la aplicación se ha condensando la información en forma de listas.

Ahora mismo no es disponible crear listas customizadas y guardarlas, por ello, se han proporcionado una serie de listas genéricas que se construyen con unos filtros fijos a la hora de instanciarse. Por ejemplo, para construir la lista de los perros puestos en adopción, se hace de la siguiente manera:

\begin{verbatim}
AvailableDogsPage(
	'Perros en adopción',
         [
		Condition('forAdoption', 'isEqualTo', true), 
		Condition('adopted', 'isEqualTo', false)
	], 
	true
)
\end{verbatim}


Estas clases disponen de un método que se encarga de concatenar todos los filtros fijos en la consulta base de la lista haciendo uso del método proporcionados por la clase \textit{Condition}.

\begin{verbatim}
getCollectionWithFilters()  {
	Query<Map<String, dynamic>> collection = FirebaseFirestore.instance.collection('dogs');
    		for (Condition condition in conditions) {
      			collection = Condition.addClauseByCondition(collection, condition);
   		}
    	return collection;
}
\end{verbatim}

Cuando se aplican filtros dinámicos, se utiliza el siguiente método que se encarga de concatenar todos los filtros rápidos aplicados a la consulta que ya contiene los filtros fijos.

\begin{verbatim}
getQuickFiltersAndSnapshot() {
    Query<Map<String, dynamic>> collection = widget.getCollectionWithFilters();
    for (Condition condition in quickFilters) {
        collection = Condition.addClauseByCondition(collection, condition);
    }
    return collection;
}
\end{verbatim}

Todas las listas de la aplicación ordenan los resultados según la distancia.

\newpage
\subsubsection*{Usuarios}

Las listas de usuarios pueden incluir tanto como usuarios normales como protectoras, algunas de las listas construidas en la aplicación contienen filtros específicos para mostrar únicamente listas de un rol deterimnado. Este tipo de listas disponen de una barra de búsqueda que filtra resultados según el texto que se introduzca, dicho texto busca coincidir con diferentes propiedades del usuario tales como la dirección o el nombre. 

\begin{figure}[H]
   	\begin{subfigure}{0.48\textwidth}
		\begin{center}
			{\includegraphics[width=6cm]{app/UserList.png}\par}
			\caption{Lista de usuarios sin búsqueda.}
		\end{center}  
	\end{subfigure}\hfill
   	\begin{subfigure}{0.48\textwidth}
		\begin{center}
			{\includegraphics[width=6cm]{app/UserListSearch.png}\par}
			\caption{Lista de usuarios con búsqueda.}
		\end{center}  
	\end{subfigure}\hfill
	\caption{Lista de usuarios.}
\end{figure}


Estas listas, además, incluyen un botón que permite ver los resultados de la lista colocados en un mapa en forma de marcador, la misma lista se colocará debajo del mapa de la página. Los marcadores pueden ser pulsados para abrir un pop-up que redirige al perfil de la protectora. Si la lista contiene usuarios sin dirección, no se añadirán al mapa.

\begin{figure}[H]
   	\begin{subfigure}{0.48\textwidth}
		\begin{center}
			{\includegraphics[width=6cm]{app/MapsPage.png}\par}
			\caption{Mapa y lista.}
		\end{center}  
	\end{subfigure}\hfill
   	\begin{subfigure}{0.48\textwidth}
		\begin{center}
			{\includegraphics[width=6cm]{app/MarkerClick.png}\par}
			\caption{Acción del marcador.}
		\end{center}  
	\end{subfigure}\hfill
	\caption{Página de mapa.}
\end{figure}

\newpage
\subsubsection*{Perros}

Las listas de perros incluyen cualquier tipo de perro que esté dado de alta en la aplicación. Los resultados dependerán de los filtros introducidos por el usuario o los definidos a la hora de construir la lista. Estas listas también incluyen una barra de búsqueda para filtrar resultados además de una barra con varios filtros rápidos que corresponden con algunas de las propiedades de los perros.

\begin{figure}[H]
   	\begin{subfigure}{0.48\textwidth}
		\begin{center}
			{\includegraphics[width=6cm]{app/DogList.png}\par}
			\caption{Lista de perros sin búsqueda o filtros.}
		\end{center}  
	\end{subfigure}\hfill
   	\begin{subfigure}{0.48\textwidth}
		\begin{center}
			{\includegraphics[width=6cm]{app/DogListSearch.png}\par}
			\caption{Lista de perros con búsqueda}
		\end{center}  
	\end{subfigure}\hfill
	\caption{Lista de perros.}
\end{figure}

\begin{figure}[H]
   	\begin{subfigure}{0.48\textwidth}
		\begin{center}
			{\includegraphics[width=6cm]{app/DogListFilter.png}\par}
			\caption{Lista de perros con filtros.}
		\end{center}  
	\end{subfigure}\hfill
   	\begin{subfigure}{0.48\textwidth}
		\begin{center}
			{\includegraphics[width=6cm]{app/DogListSearchFilter.png}\par}
			\caption{Lista de perros con búsqueda y filtros}
		\end{center}  
	\end{subfigure}\hfill
	\caption{Lista de perros con filtros.}
\end{figure}


Los widgets de los filtros abren un pop-up con las diferentes opciones disponibles cuando se pulsa en ellos. Se incluye una barra de búsqueda para filtrar las opciones disponibles.


\begin{figure}[H]
   	\begin{subfigure}{0.48\textwidth}
		\begin{center}
			{\includegraphics[width=6cm]{app/FiltersDialog.png}\par}
			\caption{Filtro con pocas opciones.}
		\end{center}  
	\end{subfigure}\hfill
   	\begin{subfigure}{0.48\textwidth}
		\begin{center}
			{\includegraphics[width=6cm]{app/FilterSearch.png}\par}
			\caption{Búsqueda en opciones de filtros.}
		\end{center}  
	\end{subfigure}\hfill
	\caption{Filtros.}
\end{figure}

% Perfiles personales
\newpage
\subsection*{Perfil personal}

A continuación se muestra la pantalla de perfil personal que se genera para cada uno de los usuarios registrados. Dependiendo de algunas condiciones se mostrarán unos botones u otros:
\begin{itemize}
	\item Si el usuario es una protectora , se incluye un botón para abrir un mapa con la protectora ubicada en él.
	\item Si el perfil que se visita no es el propio, se incluye un botón de contacto para abrir un chat con el usuario.
\end{itemize}

\begin{figure}[H]
   	\begin{subfigure}{0.48\textwidth}
		\begin{center}
			{\includegraphics[width=6cm]{app/UserProfile.png}\par}
			\caption{Perfil personal de usuario.}
		\end{center}  
	\end{subfigure}\hfill
   	\begin{subfigure}{0.48\textwidth}
		\begin{center}
			{\includegraphics[width=6cm]{app/CompanyProfile.png}\par}
			\caption{Perfil personal de protectora.}
		\end{center}  
	\end{subfigure}\hfill
	\caption{Pantalla de perfil personal.}
\end{figure}

% Actualización de datos
\newpage
\subsection*{Actualizar datos de usuario}

Los formularios para actualizar datos de usuario son casi idénticos a los proporcionados para registrarse. En primer lugar, los formularios cargan todos los datos actuales del usuario y permite que se introduzcan todos los cambios que se requieran. Posterior a confirmar los cambios, si la actualización es correcta, el usuario es redirigido a su perfil y una notifiación es mostrada indicando que la actualización se ha hecho correctamente.

\begin{figure}[H]
   	\begin{subfigure}{0.48\textwidth}
		\begin{center}
			{\includegraphics[width=6cm]{app/UserUpdate.png}\par}
			\caption{Formulario de actualización.}
		\end{center}  
	\end{subfigure}\hfill
   	\begin{subfigure}{0.48\textwidth}
		\begin{center}
			{\includegraphics[width=6cm]{app/UserUpdateSuccess.png}\par}
			\caption{Notificación de correcta actualización.}
		\end{center}  
	\end{subfigure}\hfill
	\caption{Actualización de datos personales.}
\end{figure}


% Registro perros
\newpage
\subsection*{Registro perros}

Las protectoras pueden añadir perros a la aplicación desde su lista personal llamada \textit{Mis perros}. En esta pantalla se incluye una barra superior con un botón específico que redirige al formulario de registro para perros. El formulario proporcionado incluye todos los atributos que requiere un perro para ser dado de alta.

\begin{figure}[H]
	\begin{center}
		{\includegraphics[width=6cm]{app/MyDogs.png}\par}
		\caption{Pantalla \textit{Mis perros}.}
	\end{center}  
\end{figure}


\begin{figure}[H]
   	\begin{subfigure}{0.48\textwidth}
		\begin{center}
			{\includegraphics[width=6cm]{app/RegisterDog1.png}\par}
		\end{center}  
	\end{subfigure}\hfill
   	\begin{subfigure}{0.48\textwidth}
		\begin{center}
			{\includegraphics[width=6cm]{app/RegisterDog2.png}\par}
		\end{center}  
	\end{subfigure}\hfill
	\caption{Pantalla de registro de perro.}
\end{figure}


% Perfil perros
\newpage
\subsection*{Perfil perro}

Para mostrar toda la información de un perro a un usuario, se ha condensado toda la información en una pantalla. En esta pantalla se incluye toda la información relacionada con el perro, tales como nombre, raza, peso, etc. Se incluye también un mapa que muestra la ubicación de la protectora a la que pertenece, además de un botón de contacto para abrir chat con la protectora. Si el usuario que visita el perfil es el dueño del perro, puede visualizar una sección inferior con diferentes botones para abrir la edición del perfil, eliminar el perro o marcarlo como adoptado. 

\begin{figure}[H]
   	\begin{subfigure}{0.48\textwidth}
		\begin{center}
			{\includegraphics[width=6cm]{app/DogProfile1.png}\par}
		\end{center}  
	\end{subfigure}\hfill
   	\begin{subfigure}{0.48\textwidth}
		\begin{center}
			{\includegraphics[width=6cm]{app/DogProfile2.png}\par}
		\end{center}  
	\end{subfigure}\hfill
	\caption{Pantalla de perfil de perro.}
\end{figure}

\begin{figure}[H]
   	\begin{subfigure}{0.48\textwidth}
		\begin{center}
			{\includegraphics[width=6cm]{app/AdoptedDialog.png}\par}
		\end{center}  
	\end{subfigure}\hfill
   	\begin{subfigure}{0.48\textwidth}
		\begin{center}
			{\includegraphics[width=6cm]{app/UnAdoptDog.png}\par}
		\end{center}  
	\end{subfigure}\hfill
	\caption{Pop-up de marcar/desmarcar como adoptado.}
\end{figure}


\begin{figure}[H]
	\begin{center}
		{\includegraphics[width=6cm]{app/DeleteDog.png}\par}
		\caption{Borrar perro.}
	\end{center}  
\end{figure}

\begin{figure}[H]
	\begin{center}
		{\includegraphics[width=6cm]{app/ShareAction.png}\par}
		\caption{Compartir perro.}
	\end{center}  
\end{figure}

% Actualización de datos
\newpage
\subsection*{Actualizar datos de perros}

Los formularios para actualizar datos de perros son casi idénticos a los proporcionados para darlos de alta. En primer lugar, los formularios cargan todos los datos actuales del perro y permite que se introduzcan todos los cambios que se requieran. Posterior a confirmar los cambios, si la actualización es correcta, se muestra una notificación indicándolo.

\begin{figure}[H]
   	\begin{subfigure}{0.48\textwidth}
		\begin{center}
			{\includegraphics[width=6cm]{app/DogUpdate.png}\par}
		\end{center}  
	\end{subfigure}\hfill
   	\begin{subfigure}{0.48\textwidth}
		\begin{center}
			{\includegraphics[width=6cm]{app/DogUpdate2.png}\par}

		\end{center}  
	\end{subfigure}\hfill
	\caption{Formulario de actualización.}
\end{figure}

\begin{figure}[H]
	\begin{center}
		{\includegraphics[width=6cm]{app/DogUpdateSuccesss.png}\par}
		\caption{Notificación de actualización correcta.}
	\end{center}  
\end{figure}


% Chats
\newpage
\subsection*{Chats}

La estructura de la pantalla de los chats es muy similar a cualquier otra aplicación de mensajería, se incluye una lista de chats que permiten ser borrados deslizando el widget. Cuando se hace click en alguno de estos widgets, se redirige a la pantalla del chat específico.

\begin{figure}[H]
	\begin{center}
			{\includegraphics[width=6cm]{app/ChatRoom.png}\par}
	\end{center}  
\end{figure}

\begin{figure}[H]
   	\begin{subfigure}{0.48\textwidth}
		\begin{center}
			{\includegraphics[width=6cm]{app/ChatsPage.png}\par}
		\end{center}  
	\end{subfigure}\hfill
   	\begin{subfigure}{0.48\textwidth}
		\begin{center}
			{\includegraphics[width=6cm]{app/ChatsPageReceiver.png}\par}
		\end{center}  
	\end{subfigure}\hfill
	\caption{Pantalla de chats disponibles.}
\end{figure}


% Legal Page
\newpage
\subsection*{Información legal}

La pantalla de información legal recoge los términos y condiciones de uso además de toda la información de política de privacidad de la aplicación.

\begin{figure}[H]
   	\begin{subfigure}{0.48\textwidth}
		\begin{center}
			{\includegraphics[width=6cm]{app/LegalPage.png}\par}
		\end{center}  
	\end{subfigure}\hfill
   	\begin{subfigure}{0.48\textwidth}
		\begin{center}
			{\includegraphics[width=6cm]{app/LegalPage2.png}\par}
		\end{center}  
	\end{subfigure}\hfill
	\caption{Pantalla de información legal.}
\end{figure}


% Legal Page
\newpage
\subsection*{Contacto}

En la pantalla de contacto se incluye un correo electrónico de contacto además de un botón que abre un chat con un administrador de la aplicación.

\begin{figure}[H]
	\begin{center}
		{\includegraphics[width=6cm]{app/ContactPage.png}\par}
	\end{center}  
\end{figure}

% Legal Page
\newpage
\subsection*{Notificaciones}

Las notificaciones se lanzan cuando un usuario recibe un mensaje de otro usuario.

\begin{figure}[H]
	\begin{center}
		{\includegraphics[width=6cm]{app/Notifications.png}\par}
	\end{center}  
\end{figure}


% Vista administrador
\newpage
\subsection*{Vista de administrador}

La vista de administrador es la más simple, contiene diferentes listas para manejar los usuarios y perros que están dados de alta en la aplicación. Puede editar datos de cualquier usuario o perro además de verificar y desverificar perfiles de usuario.

\begin{figure}[H]
   	\begin{subfigure}{0.48\textwidth}
		\begin{center}
			{\includegraphics[width=6cm]{app/AdminHome.png}\par}
			\caption{Pantalla inicio administrador}
		\end{center}
	\end{subfigure}\hfill
   	\begin{subfigure}{0.48\textwidth}
		\begin{center}
			{\includegraphics[width=6cm]{app/AdminDrawer.png}\par}
			\caption{Menú lateral administrador.}
		\end{center}
	\end{subfigure}\hfill
	\caption{Pantalla inicio de administrador.}
\end{figure}

\begin{figure}[H]
   	\begin{subfigure}{0.48\textwidth}
		\begin{center}
			{\includegraphics[width=6cm]{app/AdminVerifyPopUp}\par}
			\caption{Pop-up verificar.}
		\end{center}
	\end{subfigure}\hfill
   	\begin{subfigure}{0.48\textwidth}
		\begin{center}
			{\includegraphics[width=6cm]{app/AdminUnVerifyPopUp}\par}
			\caption{Pop-up desverificar.}
		\end{center}
	\end{subfigure}\hfill
	\caption{Verificar/desverificar usuario.}
\end{figure}

% Conclusiones
\newpage
\section{Conclusiones}

Tras completar el período de desarrollo de la aplicación, en este apartado se comprueba la completitud de todos los requisitos funcionales. 

Se implementó el registro de usuarios por rol además de proporcionar diferentes vistas para cada uno de los roles, que finalmente se definieron como \textit{Usuario},  \textit{Protectora}  y   \textit{Administrador}.  Con relación a los usuarios, también se añadieron diferentes listas a lo largo de la aplicación, que finalmente incluyeron una barra de búsqueda para filtrar resultados. Además, para los usuarios se implementó una página de perfil en la que se puede acceder a la edición de datos y credenciales. Por otra parte, también se implementó el registro y edición de perros y se incluyeron diferentes listas en la aplicación, que aparte de una barra de búsqueda también incluyen los filtros de  \textit{Peso},  \textit{Raza},  \textit{Género} y  \textit{Color}. Además, en los perfiles de los perros se añadieron botones para compartir y marcar como favoritos, este último vino de la mano de la implementación de la sección de favoritos para los usuarios. Se crearon diferentes páginas adicionales, como la página de contacto o la de información legal, además de la página de mapas, en la que se incluyó un listado de resultados e hizo falta integrar las APIs de geolocalización. Esta misma API de geolocalización también se utilizó para desarrollar una barra de búsqueda de direcciones. Por último, se implementó un chat en tiempo real al que se le añadió la opción de mandar imágenes con sus correspondientes notificaciones.

Al revisar la lista de objetivos, se concluye que la aplicación cumple con todas las funcionalidades obligatorias. Sin embargo, algunos requisitos opcionales no se completaron en esta versión. Específicamente, el blog y el tema oscuro no se implementaron. El blog resultó ser demasiado ambicioso y no se pudo completar en el tiempo estipulado, ni si quiera dio tiempo a definir una estructura de datos para este, mientras que el tema oscuro se pospuso debido a retrasos en la implementación de otras funcionalidades obligatorias. Estos requisitos pendientes se considerarán para futuras versiones de la aplicación.

Para futuras iteraciones, se han planteado algunas propuestas para añadir valor a la aplicación y mejorar la experiencia de uso del usuario.

\begin{itemize}[noitemsep]
	\item Añadir la posibilidad de crear listas con filtros a las protectoras.
	\item Permitir subir múltiples imágenes para los perros dados de alta, además de vídeos y poder generar un albúm por cada perro.
	\item Permitir asignar a un usuario como adoptante/acogedor de un perro, además de indicar en el perfil del perro quien es su adoptante/acogedor.
	\item Sumar las funcionalidades mandar mensajes de audios y vídeos dentro del chat
	\item Añadir como atributos opcionales de las protectoras distintas redes sociales e incluso el número de teléfono.
	\item Permitir compartir perfiles de protectoras.
	\item Integrar los servicios de Google para iniciar sesión.
	\item Incluir más filtros para las listas de caninos.
\end{itemize}

Todas estas propuestas buscan mejorar la experiencia de uso de la aplicación, algunas de ellas surgen a partir del feedback proporcionado por los usuarios que han probado la aplicación. Además, aunque la aplicación está diseñada inicialmente para facilitar las adopciones de perros, se contempla la posibilidad de incluir otros tipos de animales, como gatos, en futuras actualizaciones.


% Bibliografía
\newpage
\section{Bibliografía}
\bibliographystyle{IEEEtran}
\bibliography{referencias}


\printindex
\end{document}